\documentclass[12pt]{article}
\usepackage{fullpage}

\usepackage{amsmath}
\usepackage{amssymb}
\usepackage{amsthm, algorithm}
\usepackage{hyperref}
\usepackage{color}

\newcommand{\defeq}{\stackrel{\textup{def}}{=}}
\newcommand{\argmin}{\text{arg min}}
\newtheorem{problem}{Problem}
\newtheorem{theorem}{Theorem}[section]
\newtheorem{prop}[theorem]{Proposition}
\newtheorem{corollary}{Corollary}[theorem]
\newtheorem{remark}{Remark}[theorem]
\newtheorem{lemma}[theorem]{Lemma}
\newtheorem{definition}[theorem]{Definition}
\newtheorem{observation}[theorem]{Observation}
\newtheorem{hole}{Hole}[theorem]

\newcommand{\cupdot}{\mathbin{\mathaccent\cdot\cup}}

\newcommand{\tr}{\mbox{Trace}}
\newcommand\prob[2]{\mbox{Pr}_{#1}\left[ #2 \right]}
\newcommand\expec[2]{{\mathbb{E}}_{#1}\left[ #2 \right]}
\newcommand\var[2]{\mbox{\bf Var}_{#1}\left[ #2 \right]}


\newcommand\Ctil{\widetilde{\mathit{C}}}
\newcommand\Otil{\widetilde{\mathit{O}}}

\newcommand\Ccal{\mathcal{C}}
\newcommand\Hcal{\mathcal{H}}

\renewcommand\AA{\boldsymbol{\mathit{A}}}
\newcommand\DD{\boldsymbol{\mathit{D}}}
\newcommand\MM{\boldsymbol{\mathit{M}}}
\newcommand\MMcal{\boldsymbol{\mathcal{M}}}
\newcommand\MMbar{\boldsymbol{\overline{\mathit{M}}}}
\newcommand\MMhat{\boldsymbol{\widehat{\mathit{M}}}}
\newcommand\II{\boldsymbol{\mathit{I}}}
\newcommand\LL{\boldsymbol{\mathit{L}}}
\newcommand\LLtil{\widetilde{\boldsymbol{L}}}
\newcommand\PP{\boldsymbol{\mathit{P}}}
\newcommand\UU{\boldsymbol{\mathit{U}}}
\newcommand\XX{\boldsymbol{\mathit{X}}}
\newcommand\XXcal{\boldsymbol{\mathcal{X}}}
\newcommand\XXJcal{\boldsymbol{\mathcal{X}}\mathcal{J}}
%\newcommand\YY{\boldsymbol{\mathit{Y}}}
%\newcommand\YYcal{\mathcal{Y}}
\newcommand\ZZ{\boldsymbol{\mathit{Z}}}
\newcommand\ZZhat{\boldsymbol{\widehat{\mathit{Z}}}}

\newcommand\cchi{\boldsymbol{\chi}}

\newcommand\simuniform{{\sim_{{\rm uniform}}}}

\newcommand\Ical{\mathcal{I}}

\newcommand\yhat{\widehat{y}}

\newcommand\AbsMatrix[1]{\mbox{Abs}_{2}\left| #1 \right|}


\newcommand\dd{\boldsymbol{\mathit{d}}}
\newcommand\rr{\boldsymbol{\mathit{r}}}
\newcommand\ww{\boldsymbol{\mathit{w}}}
\newcommand\xx{\boldsymbol{\mathit{x}}}
\newcommand\yy{\boldsymbol{\mathit{y}}}
\newcommand\eps{\varepsilon}


\newcommand\PPi{\boldsymbol{\mathit{\Pi}}}
\newcommand\one{\vec{1}}

\newcommand{\todo}[1]{{\bf \color{red} TODO: #1}}
\newcommand{\richard}[1]{{\bf \color{green} RICHARD: #1}}
\newcommand{\junxing}[1]{{\bf \color{green} JUNXING: #1}}
\newcommand{\saurabh}[1]{{\bf \color{green} SAURABH: #1}}
\newcommand{\sushant}[1]{{\bf \color{green} SUSHANT: #1}}
\newcommand{\tim}[1]{{\bf \color{red} TIMOTHY: #1}}

\newcommand\LRO{\text{LRO}}

\begin{document}

\begin{theorem}
Let $D$ be a negative type matrix. Let $\LRO_k(S)$ denote the optimal
low-rank approximation of $S$ in the Froebenius norm. Then in 
$O(n \text{poly}\frac{k}{\eps})$ time, we can compute matrix $D'$ of
rank $k$ such
that:

\begin{align}
||D-D'||_F \leq (1+\eps)||D-LRO_{k-2}(D)||_F
\end{align}

\end{theorem}

To do this, we will use Musco and Woodruff and a careful use of Cauchy's
Interlacing Theorem.
\begin{theorem} (Musco and Woodruff)
  \label{thm:psd-approx}
  Let $M$ be a positive semi-definite matrix. Then we can compute a matrix
  $M'$ of rank $k$ in $O(n\text{poly}\frac{k}{\eps}$ time such that:

  \begin{align}
    ||M-M'||_F \leq (1+\eps)||M-LRO_k(M)||_F
  \end{align}
\end{theorem}
Cauchy's Interlacing theorem, as stated in
lemma 3.4 of https://arxiv.org/pdf/1408.4421v1.pdf:

\begin{theorem} (Optimal Low Rank Approximation of Matrices):
Let $\mu_1, \ldots \mu_n$ be eigenvectors of symmetric matrix $S$, where
 $|\mu_1| \leq \ldots \leq |\mu_n|$. Then:

 \[S - \LRO_k(S) = l_2(\mu_{k+1}, \ldots, \mu_n) \]
\end{theorem}

 We state the above theorem without proof. Here, $l_2$ represents the
 $l_2$ norm.
\begin{theorem} (Cauchy's Interlacing Theorem)
\label{thm:interlacing}
Let $\lambda^X_i$ denote the $i^{th}$ largest eigenvalues of $X$ for any
matrix $X$.
If $A$ is a symmetric matrix and $v$ is a vector, then the eigenvalues
of $A + vv^T$ satisfy:

\begin{align}
\lambda^A_{i} \leq \lambda^{A+vv^T}_i \leq \lambda^{A}_{i+1}
\end{align}
\begin{align}
\lambda^{A+vv^T}_{i} \leq \lambda^{A}_{i+1} \leq \lambda^{A+vv^T}_{i+1}
\end{align}

\end{theorem}

\begin{corollary}
\label{cor:rank-two-update}
\begin{align}
\lambda^A_{i-1} \leq \lambda^{A+vv^T-ww^T}_i \leq \lambda^{A}_{i+1}
\end{align}
\begin{align}
\lambda^{A+vv^T-ww^T}_{i-1} \leq \lambda^{A}_{i} \leq
\lambda^{A+vv^T-ww^T}_{i+1}
\end{align}
\end{corollary}
This follows from Theorem~\ref{thm:interlacing}.

\begin{definition}
  \label{def:}Let $M:=[d_{ij}-d_{0i}-d_{0j}]$, where $d_{ij}$ is the $ij$ entry of
$D$. Let $A_1 = [d_{0i}]$, and $A_2 = [d_{0j}]$, where $A1$ and $A2$ are
$n$ by $n$ matrices. By construction, $D = M + A_1 + A_2$, and sampling
from $M$ can be done quickly by taking three samples from $D$.

Let $v$ be the vector with entries $(1+d_{0i})/2$, and $w$ be the vector
  with entires $(1-d_{0i})/2$.
\end{definition}

\begin{lemma}
\label{lem:diff}
$A1 + A2 = vv^T - ww^T$
\end{lemma}
(Proof omitted. This is straightforward computation I think.)
\begin{lemma}
$ D = M + vv^T - ww^T $
\end{lemma}
\begin{proof}
This follows from lemma \ref{lem:diff}.
\end{proof}
\begin{corollary}
\label{cor:m-and-d}
Using $D$ and $M$ as defined in~\ref{def:}:

\begin{align}
\lambda^M_{i-1} \leq \lambda^{D}_i \leq \lambda^{M}_{i+1}
\end{align}
\begin{align}
\lambda^{D}_{i-1} \leq \lambda^{M}_{i} \leq
\lambda^{D}_{i+1}
\end{align}
      
\end{corollary}
\begin{proof}
This follows from corollary~\ref{cor:rank-two-update} and
Lemma~\ref{lem:diff}
\end{proof}

Now, we list basic facts about negative type distances $D$. 
\begin{theorem}

\[\lambda^D_1 \leq \ldots \lambda^D_{n-1} \leq 0 < \lambda_n^D\]
\[\lambda_n^D \geq |\lambda^D_1| \geq \ldots |\lambda^D_{n-1}| \geq 0 \]

\end{theorem}
The second chain of inequalities holds since $D$ has at most one
positive eigenvalue (known, I can show you how to prove this offline).
Also, the trace of $D$ is $0$, which means the positive eigenvalue is
equal to the negative sum of the negative eigenvalues.

\begin{theorem}
$M$ is negative-semidefinite, and so:
\[\lambda^M_1 \leq \ldots \leq \lambda^M_{n} = 0 \]
\[|\lambda^M_1| \geq \ldots |\lambda^M_{n}| \geq 0 \]
\end{theorem}
That $M$ is negative-semidefinite, is well established in the
literature.

\begin{theorem}
\label{thm:d-m-interlace}
For all $i < n$:
\begin{align}
|\lambda^M_{i-1}| \geq |\lambda^{D}_i| \geq |\lambda^{M}_{i+1}|
\end{align}
For all $2 \leq i < n-1$:
\begin{align}
|\lambda^{D}_{i-1}| \geq |\lambda^{M}_{i}| \geq
|\lambda^{D}_{i+1}|
\end{align}
\end{theorem}

\begin{proof}
This follows from Corollary~\ref{cor:m-and-d}.
\end{proof}

\begin{theorem} Let $M'$ be the output of Woodruff and Musco, applied to
$M$, where $M'$ is rank $k-2$. Then let $D' = M' + vv^T - ww^T$. Recall that $D = M + vv^T -
ww^T$. Then

\[ || D-D' ||_F = ||M - M'|| \leq (1+\eps)||M-LRO_{k-2}(M)|| \]
\[ = (1+\eps) \cdot l_2(\lambda_{k-1}^M, \ldots \lambda_{n}^M) \]
\end{theorem}

\begin{theorem}
For $j \geq 1$, 
\[ ||D-LRO_j(D)||_F = l_2(\lambda_j^D, \ldots \lambda_{n-1}^D) \]
\end{theorem}

\begin{theorem}
\[ l_2(\lambda_{k-1}^M, \ldots \lambda_{n}^M) 
  \leq
  l_2(\lambda_{k-2}^D, \ldots \lambda_{n-1}^D) = ||D-LRO_{k-2}(D)||_F
  \]
\end{theorem}
\begin{proof}
The first inequality follows from Theorem~\ref{thm:d-m-interlace}.
\end{proof}

\begin{theorem}
\[ ||D-D'||_F \leq (1+\eps) ||D-LRO_{k-2}(D)||_F \]
\end{theorem}
As desired. Note that I think we can actually bound $||D-D'||_F$ with
$||D-LRO_{k-1}(D)||_F$ with more careful eigenvalue bounding, but
whatevs. Note that $D'$ can be computed quickly ($M'$ can be obtained
    quickly by sampling from $D$, and $v$ and $w$ can be computed in
    $O(n)$ time.)
\end{document}
