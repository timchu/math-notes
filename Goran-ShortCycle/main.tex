\documentclass[12pt]{article}
\usepackage{fullpage}

\usepackage{amsmath}
\usepackage{amssymb}
\usepackage{amsthm, algorithm}
\usepackage{hyperref}
\usepackage{color}

\newcommand{\defeq}{\stackrel{\textup{def}}{=}}
\newtheorem{problem}{Problem}
\newtheorem{theorem}{Theorem}[section]
\newtheorem{prop}[theorem]{Proposition}
\newtheorem{corollary}{Corollary}[theorem]
\newtheorem{remark}{Remark}[theorem]
\newtheorem{lemma}[theorem]{Lemma}
\newtheorem{definition}[theorem]{Definition}
\newtheorem{observation}[theorem]{Observation}
\newtheorem{hole}{Hole}[theorem]

\newcommand{\cupdot}{\mathbin{\mathaccent\cdot\cup}}

\newcommand{\tr}{\mbox{Trace}}
\newcommand\prob[2]{\mbox{Pr}_{#1}\left[ #2 \right]}
\newcommand\expec[2]{{\mathbb{E}}_{#1}\left[ #2 \right]}
\newcommand\var[2]{\mbox{\bf Var}_{#1}\left[ #2 \right]}


\newcommand\Ctil{\widetilde{\mathit{C}}}
\newcommand\Otil{\widetilde{\mathit{O}}}

\newcommand\Ccal{\mathcal{C}}
\newcommand\Hcal{\mathcal{H}}

\renewcommand\AA{\boldsymbol{\mathit{A}}}
\newcommand\DD{\boldsymbol{\mathit{D}}}
\newcommand\MM{\boldsymbol{\mathit{M}}}
\newcommand\MMcal{\boldsymbol{\mathcal{M}}}
\newcommand\MMbar{\boldsymbol{\overline{\mathit{M}}}}
\newcommand\MMhat{\boldsymbol{\widehat{\mathit{M}}}}
\newcommand\II{\boldsymbol{\mathit{I}}}
\newcommand\LL{\boldsymbol{\mathit{L}}}
\newcommand\LLtil{\widetilde{\boldsymbol{L}}}
\newcommand\PP{\boldsymbol{\mathit{P}}}
\newcommand\UU{\boldsymbol{\mathit{U}}}
\newcommand\XX{\boldsymbol{\mathit{X}}}
\newcommand\XXcal{\boldsymbol{\mathcal{X}}}
\newcommand\XXJcal{\boldsymbol{\mathcal{X}}\mathcal{J}}
%\newcommand\YY{\boldsymbol{\mathit{Y}}}
%\newcommand\YYcal{\mathcal{Y}}
\newcommand\ZZ{\boldsymbol{\mathit{Z}}}
\newcommand\ZZhat{\boldsymbol{\widehat{\mathit{Z}}}}

\newcommand\cchi{\boldsymbol{\chi}}

\newcommand\simuniform{{\sim_{{\rm uniform}}}}

\newcommand\Ical{\mathcal{I}}

\newcommand\yhat{\widehat{y}}

\newcommand\AbsMatrix[1]{\mbox{Abs}_{2}\left| #1 \right|}


\newcommand\dd{\boldsymbol{\mathit{d}}}
\newcommand\rr{\boldsymbol{\mathit{r}}}
\newcommand\ww{\boldsymbol{\mathit{w}}}
\newcommand\xx{\boldsymbol{\mathit{x}}}
\newcommand\yy{\boldsymbol{\mathit{y}}}
\newcommand\eps{\varepsilon}


\newcommand\PPi{\boldsymbol{\mathit{\Pi}}}
\newcommand\one{\vec{1}}

\newcommand{\todo}[1]{{\bf \color{red} TODO: #1}}
\newcommand{\richard}[1]{{\bf \color{green} RICHARD: #1}}
\newcommand{\junxing}[1]{{\bf \color{green} JUNXING: #1}}
\newcommand{\saurabh}[1]{{\bf \color{green} SAURABH: #1}}
\newcommand{\sushant}[1]{{\bf \color{green} SUSHANT: #1}}
\newcommand{\tim}[1]{{\bf \color{red} TIMOTHY: #1}}

\begin{document}
Reductions; General graph to a diameter $ D = O(\log n)$ graph.

\begin{theorem}
  Low diameter decomposition: It takes linear time (single
  processor) to decompose into $O(\log n)$ size pieces, in $O(m + n\og
  n)$ time, constant fraction of edges.
\end{theorem}

Note: we only need $O(1 - 1/\log n)$ edges in cycles.

Assume small diameter: Cycle Cover. ($O(\log n)$ congestion, every edge
is covered).

\begin{theorem} Parter '18. Cycle cover doable with cycles length at
  most $D$, every edge in at most $C$ cycles. Here, $C = polylog$ and $D
  = \log$. This can be done in $O(m)$ time. (Bridgeless graphs).
\end{theorem}

Everything except time is covered in Parter '18. 
Parter '18: how to find a cycle cover. Divide the line into blocks. The
number of non-tree endpoints-of-edges is exactly $100$.  (Call this the
non-tree degree). The non-tree of a block is $50 \leq 100$ by
construction. (Can ignore over-heavy singletons).

When you contract a non-tree edge (let $M$ be non-tree edges), contract
the blocks (?), then this graph counting self loops and all that, will
have $M /50 \leq x \leq M / 100$ nodes and $M$ edges.

In this contracted graph, can find the $O(\log n)$ size cycles. Goran
has a bug!

Algorithm part 1: Take any spanning tree (of low depth). (Line not of
low depth?)

\section{Parter's Algorithm.}
For any graph, take a low diameter spanning tree by BFSing on it. Then,
contract the graph based on a pre-order traversal. Take successive
subtrees of graphs rooted at the vertex in the pre-order traversal, and
start including them in each "part" until the number of non-tree edges
leaving them is between 50 and 100.

\end{document}
