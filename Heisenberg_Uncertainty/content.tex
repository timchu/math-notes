In this note, we seek to prove Heisenberg's Uncertainty Principle
from first principles. Our primary tool is an extension of
Cauchy, which is known as the Hardy Inequality.

\begin{align}
\int_{\R^3} |\nabla \psi(x)|^2 \geq \frac{1}{4}\int_{R^3} |\psi(x)|^2/|x|^2
\end{align}
\begin{remark} Note that this is a 3 dimensional version of
Hardy's inequality, and I'm not entirely sure whether the
$\frac{1}{4}$ constant is tight. I also do not know how this
follows from Hardy's inequality on one dimensions, or if an easy
proof of the 3 dimensional case exists.
\end{remark}
Armed with this, we now consider the formulation of quantum
kinetic energy:
\begin{align}
KE := \frac{h^2}{2m}\int_{\R^3} |\nabla \psi(x)|^2.
\end{align}
Here, $h$ represents the Planck Constant, and $m$ is the mass of 
the particle. $\psi(x)^2$ is the probability distribution of the
particle at point $x$. This formulation of Kinetic energy is an
axiom.
\begin{remark} Is this formula the kinetic energy in the absence
of some force? Otherwise I think it should have some kind of
potential operator in it
\end{remark}
Now the standard deviation of the momentum of a particle is:
\begin{align}
&\sigma(p) := \sqrt{2 m\cdot KE}
\\
&= h\left(\int_{\R^3} |\nabla \psi(x)|^2\right)^{1/2}.
\end{align}
\begin{remark}
Why is this the formula for momentum? And what is the momentum,
    without taking standard deviation?
\end{remark}
Thus: we can assert:
\begin{align}
& \left(h\int_{\R^3} |\nabla \psi(x)|^2 \right)^{1/2} 
\left(\int_{\R^3} |x|^2 |\psi(x)|^2 \right)^{1/2}
  \\
& \geq \frac{h}{2}\left(\int_{\R^3}
    |\psi(x)|^2/|x|^2\right)^{1/2}
\left(\int_{\R^3} |x|^2 |\psi(x)|^2 \right)^{1/2}
  \\
& \geq \frac{h}{2} \int_{\R^3} |\psi(x)|^2
\\
& = h/2
\end{align}
where the first inequality is from Hardy, the second is from
Cauchy, and the final equality follows since $\psi(x)^2$ is a
probability distribution, and thus has integral equal to 1 over
the domain.
\begin{remark} Follow up question: When are the inequalities in
both Hardy and Cauchy tight? Are they ever?
\end{remark}
\begin{remark} Momentum is usually a function of time, so where
did that go in this equation? Schroedinger's equation promises an
answer (but there, they have some imaginary number term that is
difficult for me to comprehend). Furthermore, what kind of
distribution is stable under Schroedinger's equation?
I suspect that $|\psi(x)|$ is the complex norm of $\psi$, and is
one reason they always deal with $\psi^2$ instead of $\psi$
directly.
\end{remark}
\begin{remark}
Are there any applications of this to graph theory? This proof
suggests that the Laplacian of the grid, with mass-weights of
$\frac{1}{|x|^2}$, has constant Hardy coefficient and thus
constant eigenvalue (here $x$ is the position). This would imply
that this graph is an 'expander' of sorts, despite looking
nothing like a classical expander.
\end{remark}


