\documentclass[12pt]{article}
\usepackage{fullpage}

\usepackage{amsmath}
\usepackage{amssymb}
\usepackage{amsthm, algorithm}
\usepackage{hyperref}
\usepackage{color}

\newcommand{\defeq}{\stackrel{\textup{def}}{=}}
\newtheorem{problem}{Problem}
\newtheorem{theorem}{Theorem}[section]
\newtheorem{prop}[theorem]{Proposition}
\newtheorem{corollary}{Corollary}[theorem]
\newtheorem{remark}{Remark}[theorem]
\newtheorem{lemma}[theorem]{Lemma}
\newtheorem{definition}[theorem]{Definition}
\newtheorem{observation}[theorem]{Observation}
\newtheorem{hole}{Hole}[theorem]

\newcommand{\cupdot}{\mathbin{\mathaccent\cdot\cup}}

\newcommand{\tr}{\mbox{Trace}}
\newcommand\prob[2]{\mbox{Pr}_{#1}\left[ #2 \right]}
\newcommand\expec[2]{{\mathbb{E}}_{#1}\left[ #2 \right]}
\newcommand\var[2]{\mbox{\bf Var}_{#1}\left[ #2 \right]}


\newcommand\Ctil{\widetilde{\mathit{C}}}
\newcommand\Otil{\widetilde{\mathit{O}}}

\newcommand\Ccal{\mathcal{C}}
\newcommand\Hcal{\mathcal{H}}

\renewcommand\AA{\boldsymbol{\mathit{A}}}
\newcommand\DD{\boldsymbol{\mathit{D}}}
\newcommand\MM{\boldsymbol{\mathit{M}}}
\newcommand\MMcal{\boldsymbol{\mathcal{M}}}
\newcommand\MMbar{\boldsymbol{\overline{\mathit{M}}}}
\newcommand\MMhat{\boldsymbol{\widehat{\mathit{M}}}}
\newcommand\II{\boldsymbol{\mathit{I}}}
\newcommand\LL{\boldsymbol{\mathit{L}}}
\newcommand\LLtil{\widetilde{\boldsymbol{L}}}
\newcommand\PP{\boldsymbol{\mathit{P}}}
\newcommand\UU{\boldsymbol{\mathit{U}}}
\newcommand\XX{\boldsymbol{\mathit{X}}}
\newcommand\XXcal{\boldsymbol{\mathcal{X}}}
\newcommand\XXJcal{\boldsymbol{\mathcal{X}}\mathcal{J}}
%\newcommand\YY{\boldsymbol{\mathit{Y}}}
%\newcommand\YYcal{\mathcal{Y}}
\newcommand\ZZ{\boldsymbol{\mathit{Z}}}
\newcommand\ZZhat{\boldsymbol{\widehat{\mathit{Z}}}}

\newcommand\cchi{\boldsymbol{\chi}}

\newcommand\simuniform{{\sim_{{\rm uniform}}}}

\newcommand\Ical{\mathcal{I}}

\newcommand\yhat{\widehat{y}}

\newcommand\AbsMatrix[1]{\mbox{Abs}_{2}\left| #1 \right|}


\newcommand\dd{\boldsymbol{\mathit{d}}}
\newcommand\rr{\boldsymbol{\mathit{r}}}
\newcommand\ww{\boldsymbol{\mathit{w}}}
\newcommand\xx{\boldsymbol{\mathit{x}}}
\newcommand\yy{\boldsymbol{\mathit{y}}}
\newcommand\eps{\varepsilon}


\newcommand\PPi{\boldsymbol{\mathit{\Pi}}}
\newcommand\one{\vec{1}}

\newcommand{\todo}[1]{{\bf \color{red} TODO: #1}}
\newcommand{\richard}[1]{{\bf \color{green} RICHARD: #1}}
\newcommand{\junxing}[1]{{\bf \color{green} JUNXING: #1}}
\newcommand{\saurabh}[1]{{\bf \color{green} SAURABH: #1}}
\newcommand{\sushant}[1]{{\bf \color{green} SUSHANT: #1}}
\newcommand{\tim}[1]{{\bf \color{red} TIMOTHY: #1}}

\newcommand\ER{\textbf{r}}
\newcommand\Linv{L^{\dag}}
\newcommand\sqrtLinv{L^{\dag/2}}
\newcommand\chiuv{\chi_{uv}}
\begin{document}
\section{Introduction}
Suppose $D$ is a matrix with elements $d_{ij}$. Is it the effective resistance in
an underlying graph $G$? (Note that this paper does not account for $0$
    resistance edges, as those break the pseudoinverse propety.)

\section{Definitions}
\begin{enumerate}
\item Let $\chi_v$ denote the vector with $1$ at vertex $v$ and $0$ everywhere
else. 
\item Let $\chi_{uv}$ denote the vector $\chi_u - \chi_v$.
\item Let $P := I - \frac{1}{n}J$ where $J$ is the all ones matrix.
\item $L$ is the Laplacian of a graph $G$.
\end{enumerate}

\begin{definition}
Effective resistance between vertice $u$ and $v$ is defined to be:
$$ \ER_{uv} \defeq \chi_{uv}^T \Linv \chi_{uv}. $$
\end{definition}
where $L_{ij}$ is the \textbf{conductance} of the unique wire connecting $i$
and $j$ in underlying graph $G$. 
Note that this is the classic definition of effective resistances,
except that it only works when the resistances are non-zero.
(Conductance cannot be infinite).

\section{Is a distance ER? If so....}
\begin{align}
d_{uv} = \chiuv^T \Linv \chiuv.
\end{align}

\begin{align}
d_{uv} = \Linv_{uu} + \Linv_{vv} - 2 \cdot \Linv_{uv}.
\end{align}

\begin{align}
D = [\Linv_{uu}] + [\Linv_{vv}] - 2 \Linv.
\end{align}

\begin{align}
\frac{-P^T D P}{2} = \Linv.
\end{align}

\begin{align}
\left(\frac{-P^T D P}{2}\right)^\dag = L.
\end{align}

Or alternately, 
\begin{align}
\left(-P^T D P \right)^\dag = 2L.
\end{align}

Now we would like to check if $-P^T D P$ being Laplacian AND RANK N-1 is enough to
guarantee that $D$ is an effective resistance matrix. Here, I claim that
$-P^TDP$ is the inverse Laplacian that generates distance matrix $D$.

Proof: We claim 
\[ d_{uv} = \chiuv^T (-P^TDP/2) \chiuv. \]
Because $P \cdot \chiuv$ = $\chiuv$ since $J \cdot \chi_uv = 0$ then this claim is equivalent to:
\[ d_{uv} = \chiuv^T (-D/2) \chiuv. \]
or
\[ d_{uv} = -\frac{1}{2}(d_{uu} + d_{vv} - 2d_{uv}). \]

Since $d_{uu} = d_{vv} = 0$, then this expression is $d_{uv}$ as desired.

Note that the $P \cdot \chiuv = \chiuv$ for all $\chiuv$,
and $P$ squashing the all ones vector were
the only essential requirements on $P$. Our choice of $P$
satisfies both of these.

\end{document}
