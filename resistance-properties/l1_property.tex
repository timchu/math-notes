\documentclass[12pt]{article}
\usepackage{fullpage}

\usepackage{amsmath}
\usepackage{amssymb}
\usepackage{amsthm, algorithm}
\usepackage{hyperref}
\usepackage{color}

\newcommand{\defeq}{\stackrel{\textup{def}}{=}}
\newcommand{\argmin}{\text{arg min}}
\newtheorem{problem}{Problem}
\newtheorem{theorem}{Theorem}[section]
\newtheorem{prop}[theorem]{Proposition}
\newtheorem{corollary}{Corollary}[theorem]
\newtheorem{remark}{Remark}[theorem]
\newtheorem{lemma}[theorem]{Lemma}
\newtheorem{definition}[theorem]{Definition}
\newtheorem{observation}[theorem]{Observation}
\newtheorem{hole}{Hole}[theorem]

\newcommand{\cupdot}{\mathbin{\mathaccent\cdot\cup}}

\newcommand{\tr}{\mbox{Trace}}
\newcommand\prob[2]{\mbox{Pr}_{#1}\left[ #2 \right]}
\newcommand\expec[2]{{\mathbb{E}}_{#1}\left[ #2 \right]}
\newcommand\var[2]{\mbox{\bf Var}_{#1}\left[ #2 \right]}


\newcommand\Ctil{\widetilde{\mathit{C}}}
\newcommand\Otil{\widetilde{\mathit{O}}}

\newcommand\Ccal{\mathcal{C}}
\newcommand\Hcal{\mathcal{H}}

\renewcommand\AA{\boldsymbol{\mathit{A}}}
\newcommand\DD{\boldsymbol{\mathit{D}}}
\newcommand\MM{\boldsymbol{\mathit{M}}}
\newcommand\MMcal{\boldsymbol{\mathcal{M}}}
\newcommand\MMbar{\boldsymbol{\overline{\mathit{M}}}}
\newcommand\MMhat{\boldsymbol{\widehat{\mathit{M}}}}
\newcommand\II{\boldsymbol{\mathit{I}}}
\newcommand\LL{\boldsymbol{\mathit{L}}}
\newcommand\LLtil{\widetilde{\boldsymbol{L}}}
\newcommand\PP{\boldsymbol{\mathit{P}}}
\newcommand\UU{\boldsymbol{\mathit{U}}}
\newcommand\XX{\boldsymbol{\mathit{X}}}
\newcommand\XXcal{\boldsymbol{\mathcal{X}}}
\newcommand\XXJcal{\boldsymbol{\mathcal{X}}\mathcal{J}}
%\newcommand\YY{\boldsymbol{\mathit{Y}}}
%\newcommand\YYcal{\mathcal{Y}}
\newcommand\ZZ{\boldsymbol{\mathit{Z}}}
\newcommand\ZZhat{\boldsymbol{\widehat{\mathit{Z}}}}

\newcommand\cchi{\boldsymbol{\chi}}

\newcommand\simuniform{{\sim_{{\rm uniform}}}}

\newcommand\Ical{\mathcal{I}}

\newcommand\yhat{\widehat{y}}

\newcommand\AbsMatrix[1]{\mbox{Abs}_{2}\left| #1 \right|}


\newcommand\dd{\boldsymbol{\mathit{d}}}
\newcommand\rr{\boldsymbol{\mathit{r}}}
\newcommand\ww{\boldsymbol{\mathit{w}}}
\newcommand\xx{\boldsymbol{\mathit{x}}}
\newcommand\yy{\boldsymbol{\mathit{y}}}
\newcommand\eps{\varepsilon}


\newcommand\PPi{\boldsymbol{\mathit{\Pi}}}
\newcommand\one{\vec{1}}

\newcommand{\todo}[1]{{\bf \color{red} TODO: #1}}
\newcommand{\richard}[1]{{\bf \color{green} RICHARD: #1}}
\newcommand{\junxing}[1]{{\bf \color{green} JUNXING: #1}}
\newcommand{\saurabh}[1]{{\bf \color{green} SAURABH: #1}}
\newcommand{\sushant}[1]{{\bf \color{green} SUSHANT: #1}}
\newcommand{\tim}[1]{{\bf \color{red} TIMOTHY: #1}}

\begin{document}
We prove Effective Resistance isometrically embeds into $L1$, completing
our exploration on where in the Deza heirarchy of metrics that effective
resistance fits into. The proof is an inductive argument using Sherman
Morrison.
Throughout this note, let $\chi_{ij}$ is defined as
the vector with $1$ at vertex $i$ and $-1$ at vertex $j$ and zero
elsewhere.

\begin{theorem} Effective Resistance is in $L1$. \end{theorem}

We prove this using two theorems. Let $R_e$ be the effective resistance
of edge $e$, and $c_e$ its conductance.

\begin{theorem}
For any graph $G$,
let $\mathcal{G}'$ be the set of graphs with one edge removed from
$G$. 

\[ L_G^\dag = (m-n+1) \cdot 
\sum_{e \in \mathcal{E(G)}'}
\lim_{c \rightarrow 1}
(1-c \cdot R_ec_e) \cdot L_{G
- c \cdot e}^\dag \]
\end{theorem}
How does this help us? Well, if $G$'s minimum edge separator has a
cardinality of more than $1$, then the
above equation is equivalent to:
\[
  L_G^\dag = (m-n+1) \cdot 
\sum_{e \in \mathcal{E(G)}'}
(1- R_ec_e) \cdot L_{G
- e}^\dag
\]
and thus:
\[ \chi_{ij}^T L_G^\dag \chi_{ij} =
(m-n+1) \cdot 
\sum_{e \in \mathcal{E(G)}'}
\chi_{ij}^T\left(
(1- R_ec_e) \cdot L_{G
- e}^\dag \right) \chi_{ij} \]
and we are done by induction on the edges in $G$.

\section{If $e$ is an edge-separator}

Note that if $e$ is not an edge separator of $G$, then
\[
\lim_{c \rightarrow 1}
(1-c \cdot R_ec_e) \cdot L_{G
- c \cdot e}^\dag
\]
is equal to:
\[
(1- \cdot R_ec_e) \cdot L_{G - e}^\dag.
\]
Now we examine what happens if $e$ is an edge-separator of $G$.

\begin{theorem}
If $e$ is an edge-separator of $G$ that splits $G$ into two clusters,
then:
\[
\chi_{ij}^T \left(\lim_{c \rightarrow 1}
(1-c \cdot R_ec_e) \cdot L_{G
- c \cdot e}^\dag\right) \chi_ij
\]
is equal to $0$ if $i$ and $j$ are in the same cluster, and $c_e$
otherwise.
\end{theorem}

\end{document}
