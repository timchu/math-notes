\documentclass[12pt]{article}
\usepackage{fullpage}

\usepackage{amsmath}
\usepackage{amssymb}
\usepackage{amsthm, algorithm}
\usepackage{hyperref}
\usepackage{color}

\newcommand{\defeq}{\stackrel{\textup{def}}{=}}
\newcommand{\argmin}{\text{arg min}}
\newtheorem{problem}{Problem}
\newtheorem{theorem}{Theorem}[section]
\newtheorem{prop}[theorem]{Proposition}
\newtheorem{corollary}{Corollary}[theorem]
\newtheorem{remark}{Remark}[theorem]
\newtheorem{lemma}[theorem]{Lemma}
\newtheorem{definition}[theorem]{Definition}
\newtheorem{observation}[theorem]{Observation}
\newtheorem{hole}{Hole}[theorem]

\newcommand{\cupdot}{\mathbin{\mathaccent\cdot\cup}}

\newcommand{\tr}{\mbox{Trace}}
\newcommand\prob[2]{\mbox{Pr}_{#1}\left[ #2 \right]}
\newcommand\expec[2]{{\mathbb{E}}_{#1}\left[ #2 \right]}
\newcommand\var[2]{\mbox{\bf Var}_{#1}\left[ #2 \right]}


\newcommand\Ctil{\widetilde{\mathit{C}}}
\newcommand\Otil{\widetilde{\mathit{O}}}

\newcommand\Ccal{\mathcal{C}}
\newcommand\Hcal{\mathcal{H}}

\renewcommand\AA{\boldsymbol{\mathit{A}}}
\newcommand\DD{\boldsymbol{\mathit{D}}}
\newcommand\MM{\boldsymbol{\mathit{M}}}
\newcommand\MMcal{\boldsymbol{\mathcal{M}}}
\newcommand\MMbar{\boldsymbol{\overline{\mathit{M}}}}
\newcommand\MMhat{\boldsymbol{\widehat{\mathit{M}}}}
\newcommand\II{\boldsymbol{\mathit{I}}}
\newcommand\LL{\boldsymbol{\mathit{L}}}
\newcommand\LLtil{\widetilde{\boldsymbol{L}}}
\newcommand\PP{\boldsymbol{\mathit{P}}}
\newcommand\UU{\boldsymbol{\mathit{U}}}
\newcommand\XX{\boldsymbol{\mathit{X}}}
\newcommand\XXcal{\boldsymbol{\mathcal{X}}}
\newcommand\XXJcal{\boldsymbol{\mathcal{X}}\mathcal{J}}
%\newcommand\YY{\boldsymbol{\mathit{Y}}}
%\newcommand\YYcal{\mathcal{Y}}
\newcommand\ZZ{\boldsymbol{\mathit{Z}}}
\newcommand\ZZhat{\boldsymbol{\widehat{\mathit{Z}}}}

\newcommand\cchi{\boldsymbol{\chi}}

\newcommand\simuniform{{\sim_{{\rm uniform}}}}

\newcommand\Ical{\mathcal{I}}

\newcommand\yhat{\widehat{y}}

\newcommand\AbsMatrix[1]{\mbox{Abs}_{2}\left| #1 \right|}


\newcommand\dd{\boldsymbol{\mathit{d}}}
\newcommand\rr{\boldsymbol{\mathit{r}}}
\newcommand\ww{\boldsymbol{\mathit{w}}}
\newcommand\xx{\boldsymbol{\mathit{x}}}
\newcommand\yy{\boldsymbol{\mathit{y}}}
\newcommand\eps{\varepsilon}


\newcommand\PPi{\boldsymbol{\mathit{\Pi}}}
\newcommand\one{\vec{1}}

\newcommand{\todo}[1]{{\bf \color{red} TODO: #1}}
\newcommand{\richard}[1]{{\bf \color{green} RICHARD: #1}}
\newcommand{\junxing}[1]{{\bf \color{green} JUNXING: #1}}
\newcommand{\saurabh}[1]{{\bf \color{green} SAURABH: #1}}
\newcommand{\sushant}[1]{{\bf \color{green} SUSHANT: #1}}
\newcommand{\tim}[1]{{\bf \color{red} TIMOTHY: #1}}

\begin{document}

This note contains an observation about characters in
representation theory, and conjectures a link between this, QM,
and the theory of graph laplacians or PSD matrices in general.

Section~\ref{sec:summary} contains a summary of Fulton Harris chapters $1$ and $2$,
according to Tim Chu (Orthogonality of Characters via dimension
of center of Tensors, which occurs via Schur's lemma and the idea
of $G$-linear maps).

\subsection{Character Tables as Eigenvectors?}
Character tables are orthogonal, and it turns out that
$(P^{*}DP)_{ij}$ ($D$ diagonal with elements $a_i$, corresponding
to characters of representation $A$) is equal to: 

(a) the size of $(V_i \tensor A \tensor V_j^{*})_G$, AKA the size
of the center of $(V_i \tensor A \tensor V_j^{*})$ 

(b) the $V_i$ term in the irreducible decomposition of $(A
\tensor V_j)$.

Here, $P$ is the unitary matrix with elements $\chi_{V_j}(C_i)$,
as element $ij$, where $V_j$ is the $j^{th}$ irreducible
representation of $G$ and $C_i$ is the $i^{th}$ conjugacy class
for the group $g$. The enumeration of $C$ and $V$ is arbitrary.

Here, $P^{*}DP$ is a matrix with eigenvectors equal to the
columns of $P$, with eigenvalues $D$ (at least in the case of
reals. What happens for complex numbers?) 

\section{Reason For Considering This}
The reason I bring this up is that:

(a) Magically you get unitary matrices from representation
characters, just
like you get unitary matrices when doing an eigen-decomposition
of any PSD matrix.

(b) Fourier series can be interpreted as projecting onto the
eigenspectrum of the discrete Laplacian, and Fourier transforms
can be interpreted(-ish) as projections onto the continuous
circle (a compact set, which may be important). However,
imaginary numbers don't necessarily come up here -- you can get
away with decompositions into sine and cosine. (This is sorta why
group representations may be fundamentally different than Laplacian eigenvectors --
most Laplacians don't have complex eigenvectors.)

(c) Fourier series can also be interpreted as projecting onto the
character of the cyclic group.

\vspace{3 mm}
Some possible other throwaway reasons for considering this
include: 

(d) Magically, there is a unique Hermetian inner product for any
irreducible representation $V$, such that the representation of all
group elements (in $V$) are unitary matrices under this inner product (Last exercise of Chp 1 in Fulton
Harris). This is something I haven't looked into yet.

(e) Unitary matrices appear in Quantum Mechanics, and orthonormal
matrices appear when decomposing any positive semi-definite
matrix. (Unitary appears when decomposing a Hermitian matrix.)

(f) There exists an extension of Fourier analysis to non-abelian
groups, via character theory. Probably there is some 'Fourier analysis of semi-simple
Abelian groups' too or something (used in Terry Tao's proof of
the Froebenius Kernel Theorem, some theorem in which characters
are essential but I don't know what it says).

(g) PSD matrices appear all over TCS, and appear in the study of
negative type distances.

\section{Orthogonality of Characters Note}\label{sec:summary}
For proof on the orthogonality of characters, refer to
Fulton-Harris. The main idea is that inner products in characters
of $V$ and $W$
is equal to: the dimension of the fixed points of $V^{*} \tensor W$
under $G$, aka the \textbf{center} of $G$ in $V^{*} \tensor W$
(for any representations $V$ and $W$). (Why is inner product
equal to the center's dimension? Why does the center's dimension
imply orthogonality?)

\textbf{A note on characters, and proof that the center is
trivial for $V^{*} \tensor W$ on irreducible non-iso $V$ and $W$}

Let $g_{V^{*}}$ have eigenvalues $\alpha_1 \ldots \alpha_m$, and
$g_{W}$ have eigenvalues $\beta_1 \ldots \beta_n$. Then $g_{V^{*}
\tensor W}$ has eigenvalues $\alpha_i \beta_j$, whose sum is
$(\sum_{i} \alpha_i)(\sum_j \beta_j)$. This implies that
\[ \chi_{V^{*} \tensor W}(g) = \chi_{V^{*}}(g) \chi_{W}(g). \]
However, the center of $V^{*} \tensor W$ is equal to the
dimension of the space of $G$-linear maps (maps
$\psi: V \rightarrow W$ such that $\psi(g_V \_) = g_W \psi(\_)$) from $V$ to $W$, which by Schur's
lemma is $0$ if $V$ and $W$ are irreducible and non-isomorhpic,
and $1$ otherwise (up to some scaling, which is important but
which I'm too lazy to think about).

\textbf{Schur's Lemma:}

Schur's lemma holds since the kernel of $\psi$ is a
sub-representation of $V$, and the image of $\psi$ is a
sub-representation of $W$. (This follows from the definition of a
$G$-linear map, and in fact illustrates why the concept of a
$G$-linear map is so important.) So if $V$ and $W$ are irreducible
(have no sub-representation), then either $\psi$ is an
isomorphism, or its the zero map.

(Schur part 2) Isomorphisms must be scalar copies of the
identity:

For any $G$-linear map $\psi$ (why do we care about linearity
again? I see why we care about the $G$-preservation part),
it has an eigenvalue $\lambda$. Then $\psi-\lambda I$ is also a
$G$-linear map, since $G$-linear maps are closed under linear
combinations (why?). But then $\psi - \lambda I$ must be $0$, by
the first part of Schur's lemma. 

Note that $V$ and $W$ are treated as vector spaces over the
complex numbers, which critically are algebraically complete
(otherwise, $\psi$ might not have an eigenvalue). I don't really
understand this part, and it would be worth going into detail
about this at some point.


\subsection{Tensors Blah} 
Recall that the linear transform $g$ operating on $V^{*} \tensor W$,
has eigenvectors $EigVec(g_v) \tensor EigVec(g_w)$ with
eigenvalues $EigVal(g_v) \cdot EigVal(g_w)$. (This is crappy
notation.)

As a reminder, recall that 
\[ g_{V \tensor W} (\sum_{i} v_i \tensor w_i ) = \sum_i (g_V
\cdot v_i) \tensor (g_W \cdot w_i) \]

\subsection{Miscellaneous Notes}
As a final note, it's still unclear to me how characters are
really magical. I suspect they can't really be that strong -- the
character orthogonality is just disguising Schur's lemma, and
some properties of tensors. Somehow they don't \textit{quite}
seem like the right fundamental structure.....
\section{Contributors}
Observation about $P^{*}DP$ thanks to Ben Gunby.
\end{document}
