\documentclass[12pt]{article}
\usepackage{fullpage}

\usepackage{amsmath}
\usepackage{amssymb}
\usepackage{amsthm, algorithm}
\usepackage{hyperref}
\usepackage{color}

\newcommand{\defeq}{\stackrel{\textup{def}}{=}}
\newcommand{\argmin}{\text{arg min}}
\newtheorem{problem}{Problem}
\newtheorem{theorem}{Theorem}[section]
\newtheorem{prop}[theorem]{Proposition}
\newtheorem{corollary}{Corollary}[theorem]
\newtheorem{remark}{Remark}[theorem]
\newtheorem{lemma}[theorem]{Lemma}
\newtheorem{definition}[theorem]{Definition}
\newtheorem{observation}[theorem]{Observation}
\newtheorem{hole}{Hole}[theorem]

\newcommand{\cupdot}{\mathbin{\mathaccent\cdot\cup}}

\newcommand{\tr}{\mbox{Trace}}
\newcommand\prob[2]{\mbox{Pr}_{#1}\left[ #2 \right]}
\newcommand\expec[2]{{\mathbb{E}}_{#1}\left[ #2 \right]}
\newcommand\var[2]{\mbox{\bf Var}_{#1}\left[ #2 \right]}


\newcommand\Ctil{\widetilde{\mathit{C}}}
\newcommand\Otil{\widetilde{\mathit{O}}}

\newcommand\Ccal{\mathcal{C}}
\newcommand\Hcal{\mathcal{H}}

\renewcommand\AA{\boldsymbol{\mathit{A}}}
\newcommand\DD{\boldsymbol{\mathit{D}}}
\newcommand\MM{\boldsymbol{\mathit{M}}}
\newcommand\MMcal{\boldsymbol{\mathcal{M}}}
\newcommand\MMbar{\boldsymbol{\overline{\mathit{M}}}}
\newcommand\MMhat{\boldsymbol{\widehat{\mathit{M}}}}
\newcommand\II{\boldsymbol{\mathit{I}}}
\newcommand\LL{\boldsymbol{\mathit{L}}}
\newcommand\LLtil{\widetilde{\boldsymbol{L}}}
\newcommand\PP{\boldsymbol{\mathit{P}}}
\newcommand\UU{\boldsymbol{\mathit{U}}}
\newcommand\XX{\boldsymbol{\mathit{X}}}
\newcommand\XXcal{\boldsymbol{\mathcal{X}}}
\newcommand\XXJcal{\boldsymbol{\mathcal{X}}\mathcal{J}}
%\newcommand\YY{\boldsymbol{\mathit{Y}}}
%\newcommand\YYcal{\mathcal{Y}}
\newcommand\ZZ{\boldsymbol{\mathit{Z}}}
\newcommand\ZZhat{\boldsymbol{\widehat{\mathit{Z}}}}

\newcommand\cchi{\boldsymbol{\chi}}

\newcommand\simuniform{{\sim_{{\rm uniform}}}}

\newcommand\Ical{\mathcal{I}}

\newcommand\yhat{\widehat{y}}

\newcommand\AbsMatrix[1]{\mbox{Abs}_{2}\left| #1 \right|}


\newcommand\dd{\boldsymbol{\mathit{d}}}
\newcommand\rr{\boldsymbol{\mathit{r}}}
\newcommand\ww{\boldsymbol{\mathit{w}}}
\newcommand\xx{\boldsymbol{\mathit{x}}}
\newcommand\yy{\boldsymbol{\mathit{y}}}
\newcommand\eps{\varepsilon}


\newcommand\PPi{\boldsymbol{\mathit{\Pi}}}
\newcommand\one{\vec{1}}

\newcommand{\todo}[1]{{\bf \color{red} TODO: #1}}
\newcommand{\richard}[1]{{\bf \color{green} RICHARD: #1}}
\newcommand{\junxing}[1]{{\bf \color{green} JUNXING: #1}}
\newcommand{\saurabh}[1]{{\bf \color{green} SAURABH: #1}}
\newcommand{\sushant}[1]{{\bf \color{green} SUSHANT: #1}}
\newcommand{\tim}[1]{{\bf \color{red} TIMOTHY: #1}}

\begin{document}
\section{Abstract}
We show that effective resistance satisfies, for any integer vector $v$
whose coordinate sum is $1$:

\[
  v^T D v \leq 0.
  \]
In other words, we will show effective resistance is a Hypermetric.

This project is part of a larger goal: we'd like to determine where in the heirarchy of
metrics Effective Resistance sits. The heirarchy in question is:

\[ l_2 \subset Sphere \subset l_1 \subset HyperMetric \subset Neg-Type. \]

Note that we can show effective resistance is not contained in $l_2$
(trees are in $ER$), that $l_2$ is not contained in effective resistance
(the Euclidean square cannot embed). It was known that effective
resistances were negative type. We still do not know if effective
resistances are in $l_1$.
\section{Proof Overview}
Suppose a graph $G$ has positive conductances $c_e$ and Laplacian $L$. Let $R_e$
denote the effective resistance of edge $e$. For any
weights $w_e \subset \mathbb{R}$:
\[ \sum w_e R_e \]
can be reduced (or kept at the same quantity) by taking any edge $e'$
and either raising $c_{e'}$ to be arbitrarily high, or zero. Note that
$v^T D v$ can be written as $\sum w_e R_e$ for weights $w_{ij} =
v_{i}v_{j}$. (Proof omitted).

Consider any tree on $G$, where each tree edge has non-zero
conductance. (There must exist such a tree if $G$ is connected by
    resistance wires). For every non-tree edge, we either raise the
conductance to infinity, or we zero the conductance: whichever one happens to
increase $\sum w_e R_e$. If we zero an edge out, we reduce the vertex
count. If we zero the conductance of an edge, we keep going until
we've processed all the non-tree edges.

Now if we look at the conductances in the graph, we are either left with
a tree (if we zero-ed out all non-tree edges),
     on which the hyper-metric property
holds (trees are in $l_1$, which are hypermetrics), or we are left with
an infinite-conductance edge. If we are left with a tree, we are finished:
we raised $v^T D v$ by zeroing out non-tree edges, and we were left with
a quantity that was $\leq 0$. 

I claim that if we zero the conductance of an edge $ij$, we can finish
by induction on the number of vertices. I omit the proof here, and will
show it on the blackboard.

\section{Filling in the Proofs for the Outline.}
Let $\chi_{ij}$ be the vector that is $1$ at $i$, and $-1$ at $j$, and zero
elsewhere. Recall that $R_{ij} := \chi_{ij}^T L^\dag \chi_{ij}$.
\begin{theorem}
Let $R_e$ denote the effective resistance of edge $e$ in some graph $G$
with non-negative conductacnes. For any real weights $w_e'$ and any edge
$e$, the expression $\sum_e w_e R_e$ is monotonic in $c_{e'}$,
\end{theorem}
\begin{proof}
Sherman-Morrison for pseudoinverses tells us that:

\[ (M + u^T u)^\dag = M^\dag - \frac{M^\dag u u^T M^\dag}{1 + u^T M^\dag u}
\]
Note that raising $c_{e'}$ changes the Laplacian by a rank one update.
(Proof omitted). Let $L'$ be the Laplacian $L$ after raising $c_{e'}$ by
scalar $k$, where $k$ can be positive or negative. Now, $u$ denotes the
vector $\chi_e$.  Then:

\[ (L')^\dag = (L + ku^T u)^\dag = L^\dag -
\frac{k \cdot L^\dag u u^T L^\dag}{1 + k \cdot u^T L^\dag u}
\]
Therefore the change in effective resistance (using the standard
    effective resistance formula) is:

\[ \sum w_e (R_e - R'_e) \]
\[ = \sum w_e \chi_e^T \left((L')^\dag - L^\dag\right) \chi_e \]
\[ = \sum -w_e \left(\chi_e^T \frac{k \cdot L^\dag u u^T L^\dag}{1 + k \cdot
  u^T L^\dag u} \chi_e\right)
 \]
\begin{align}
\label{eq:er-change}
= \sum_e \frac{k}{1+k\cdot u^T L^\dag u}
\left(-w_e \cdot 
    \chi_e^T L^\dag u u^T M^\dag \chi_e\right).
 \end{align}

 Letting 
 \[C \defeq u^T L^\dag u\]
 \[S_e \defeq - w_e \cdot \chi_e^T L^\dag u u^T L^\dag \chi_e, \]
 Equation~\ref{eq:er-change} becomes:

 \begin{align}
 &\sum_e \frac{k \cdot S_e }{1+k \cdot C}
 \\
 &= \frac{k \cdot \sum_e S_e }{1+k \cdot C}
 \\
 &= \frac{k \cdot T}{1+k \cdot C}
 \end{align}
 where $T := \sum S_e$.

 This is a monotonic function in $k$ (proof omitted). Therefore, the extremes occur when
 $k$ is set to be as negative as possible (zero-ing out $k$), or by
 raising it to be as large as possible (setting it to be arbitrarily
     large).

 (Note: I'm not actually sure Sherman Morrison is true for
  Pseudoinverses.... or what conditions need to be set on the rank one
  update. Perhaps it suffices that the rank one update has a nullspace
  inside the original matrix $M$'s nullspace?)
\end{proof}
This completes the proof of our main lemma. The rest of our proof
follows the Proof Overview. (Detailed omitted.)
\end{document}
