There exist expanders without densifiers. In particular, there
exist graphs $G$ satisfying:

\[ \frac{1}{\text{poly log}} K_n \leq G \leq K_n \]

that do not admit graphs with edge weights $ \leq 10 \text{poly
log} /n$ that preserve all cuts up to a constant factor
approximation. (Such a graph is called a densifier). Here, $K_n$
is the complete graph on $n$ vertices.

\textbf{Construction:}
Construct a clique with vertices labeled $1$ through $a$. Construct a
second clique with vertices $1'$ through $a'$. Each clique has
edge weights $\frac{1}{\log n}$. Now build a matching of weight
$a$ between vertices $k$ and $k'$ for all $k$. Such a graph has
$2a$ vertices. This is a graph with a heavy matching between two
cliques. Now set $a = n/2$ and let this graph be denoted as
$Q_n$.

Basic lemmas about densifier:
\begin{lemma} 
  \begin{enumerate}
    \item If $H$ is a densifier of $G$, then $T(H)$ is a
      densifier of $G$, where $T$ is any graph isomorphism of
      $G$.
    \item If $H_1$ and $H_2$ are both densifiers of $G$, then
      any weighted average of $H_1$ and $H_2$ are densifiers of
      $G$.
    \item If $\textbf{T}$ is a family of isomorphisms of $G$ and
      $H$ is a densifier, then the weighted average 
      \[ \sum_{T \in \textbf{T}} T(H) \] is also a densifier of
      $G$.
  \end{enumerate}
\end{lemma}
I think each item is straightforward -- correct me if I'm wrong.

\begin{theorem} $Q_n$ is an expander with poly log expansion \end{theorem}
  \begin{proof} Every cut has conductance between $O(n / log n)$
    and $O(n)$
  \end{proof}

Now our main theorem:
\begin{theorem} $Q_n$ (defined in our construction) has no
densifier \end{theorem}
\begin{proof} If $Q_n$ has a $2$-densifier, then it has a densifier
  whose clique edge weights are all the same. This can be
  obtained by averaging over all graph isomorphisms that permute
  the clique vertices. 

  Moreover, the edge weights on the cliques are no more than
  twice the original edge weights, and no less than half the
  original edge weights. This is because the cut defined by
  vertices $1, 1', 2, 2', \ldots a/2, a/2'$ has cut value of $a^2 /2
  \log n$, as it cuts no edges in the matching. 

  Thus, it follows that the cliques must have edge weights no
  more than double and no less than half of their original edge
  weights.

  Now, by averaging over all permutations of vertices that fix the matching,
  you can similarly show that there is a densifier where edge
  weights on edge $ij'$ are all the same, for $i\not=j$. These
  edge weights can't be too small, as the degree of each $i$ must
  be preserved, weight on $ii'$ cannot be large (thus the edge
  weight previously on $ii'$ must be distributed across edges
  $ij'$ for $j\not=i$).

  This ensures that the cut $1, 1', 2, 2', \ldots a/2, a/2'$ must
  have cut value of $a^2/2$, which is at least a $\log n$
  distortion from the original cut value. (I skipped a couple steps in this
  deduction -- check me on this.)
\end{proof}



