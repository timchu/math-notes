First, we prove that $L^{1/2}$ is a Laplacian.

\[ L = (D-A) \]
\[ (D-A)^{1/2} = D^{1/4} (1-\hat{A})^{1/2} D^{1/4} \]
Now we sub $\hat{A} = A$ and ignore the D terms.
\[ (I-A)^{1/2} = I - A/2 - A^2/8 - A^3/16 - \ldots \]
where $p(A)$ is an degree-weighted adjacency matrix for any
positive polynomial $p$.  It's clear that $L^{1/2}$ has the same
nullspace as $L$, and half powers are possible given the matrix
is PSD. Thus, the result has negativ off-diagonals, positive
diagonals (why?), and has the right null-space: so it must be
Laplacian.
 
The core insight is the following: 
\[D^{-1/2} A D^{-1/2}\]
is an adjacency matrix, and so is any power thereof.

We seek to emulate this strategy for general matrices. For
instance, suppose $M$ is an $l_1$ covariance matrix. Now we
consider:
\[ \widehat{A} = D_M^{-1/2} A_M D_M^{-1/2} \] where $A_M$ is the off-diagonal
portion and $D_M$ is the on-diagonal portion.
Particularly, we would like to test whether:
\[ D_M^{1/2} (I-\widehat{A}^p)D_M^{1/2} \]  is an $l_1$
covariance matrix.

