Finite point sets equipped with a given negative type distance can be
embedded as points in Euclidean space, where the distance squared
between points matches the given metric. It is known that $l_1$,
$l_2$, effective resistance, hypermetrics, spherical metrics, and
more are of negative type.

For points in Euclidean space, we can calculate the covariance
matrix. It is known that if the distance in question is an
effective resistance distance, then the covariance matrix is a
pseudo-inverse Laplacian. It is also known that any power $0 < p < 1$
of the pseudoinverse Laplacian gives rise to a new pseudoinverse
Laplacian. 

Therefore, one may ask the question: 
given an $l_1$ covariance matrix, what
powers of it are still $l_1$ covariance matrices? Likewise for
$l_2$ covariance matrices, and negative type covariance matrices
in general.

Experimental evidence indicates that: taking the $p$ power of
any covariance matrix ($l_1, l_2$) fixes the class, for $0 < p <
1$. We aim to prove it, and hope that our results have
applications in... something.
