\documentclass[12pt]{article}
\usepackage{fullpage}

\usepackage{amsmath}
\usepackage{amssymb}
\usepackage{amsthm, algorithm}
\usepackage{hyperref}
\usepackage{color}

\newcommand{\defeq}{\stackrel{\textup{def}}{=}}
\newcommand{\argmin}{\text{arg min}}
\newtheorem{problem}{Problem}
\newtheorem{theorem}{Theorem}[section]
\newtheorem{prop}[theorem]{Proposition}
\newtheorem{corollary}{Corollary}[theorem]
\newtheorem{remark}{Remark}[theorem]
\newtheorem{lemma}[theorem]{Lemma}
\newtheorem{definition}[theorem]{Definition}
\newtheorem{observation}[theorem]{Observation}
\newtheorem{hole}{Hole}[theorem]

\newcommand{\cupdot}{\mathbin{\mathaccent\cdot\cup}}

\newcommand{\tr}{\mbox{Trace}}
\newcommand\prob[2]{\mbox{Pr}_{#1}\left[ #2 \right]}
\newcommand\expec[2]{{\mathbb{E}}_{#1}\left[ #2 \right]}
\newcommand\var[2]{\mbox{\bf Var}_{#1}\left[ #2 \right]}


\newcommand\Ctil{\widetilde{\mathit{C}}}
\newcommand\Otil{\widetilde{\mathit{O}}}

\newcommand\Ccal{\mathcal{C}}
\newcommand\Hcal{\mathcal{H}}

\renewcommand\AA{\boldsymbol{\mathit{A}}}
\newcommand\DD{\boldsymbol{\mathit{D}}}
\newcommand\MM{\boldsymbol{\mathit{M}}}
\newcommand\MMcal{\boldsymbol{\mathcal{M}}}
\newcommand\MMbar{\boldsymbol{\overline{\mathit{M}}}}
\newcommand\MMhat{\boldsymbol{\widehat{\mathit{M}}}}
\newcommand\II{\boldsymbol{\mathit{I}}}
\newcommand\LL{\boldsymbol{\mathit{L}}}
\newcommand\LLtil{\widetilde{\boldsymbol{L}}}
\newcommand\PP{\boldsymbol{\mathit{P}}}
\newcommand\UU{\boldsymbol{\mathit{U}}}
\newcommand\XX{\boldsymbol{\mathit{X}}}
\newcommand\XXcal{\boldsymbol{\mathcal{X}}}
\newcommand\XXJcal{\boldsymbol{\mathcal{X}}\mathcal{J}}
%\newcommand\YY{\boldsymbol{\mathit{Y}}}
%\newcommand\YYcal{\mathcal{Y}}
\newcommand\ZZ{\boldsymbol{\mathit{Z}}}
\newcommand\ZZhat{\boldsymbol{\widehat{\mathit{Z}}}}

\newcommand\cchi{\boldsymbol{\chi}}

\newcommand\simuniform{{\sim_{{\rm uniform}}}}

\newcommand\Ical{\mathcal{I}}

\newcommand\yhat{\widehat{y}}

\newcommand\AbsMatrix[1]{\mbox{Abs}_{2}\left| #1 \right|}


\newcommand\dd{\boldsymbol{\mathit{d}}}
\newcommand\rr{\boldsymbol{\mathit{r}}}
\newcommand\ww{\boldsymbol{\mathit{w}}}
\newcommand\xx{\boldsymbol{\mathit{x}}}
\newcommand\yy{\boldsymbol{\mathit{y}}}
\newcommand\eps{\varepsilon}


\newcommand\PPi{\boldsymbol{\mathit{\Pi}}}
\newcommand\one{\vec{1}}

\newcommand{\todo}[1]{{\bf \color{red} TODO: #1}}
\newcommand{\richard}[1]{{\bf \color{green} RICHARD: #1}}
\newcommand{\junxing}[1]{{\bf \color{green} JUNXING: #1}}
\newcommand{\saurabh}[1]{{\bf \color{green} SAURABH: #1}}
\newcommand{\sushant}[1]{{\bf \color{green} SUSHANT: #1}}
\newcommand{\tim}[1]{{\bf \color{red} TIMOTHY: #1}}

\begin{document}
A short probability puzzle:

Suppose you have two bags, A and B. There is a probability
distribution over A and B, on where one of $n$ toys should go. Let the
toy go in bag $A$ with probability $a$, and bag $B$ with probability
$b$.

The objective is: if only $t$ toys are placed in all bags, then bucket $B$ should
concentrate around $\eps$ of what you'd expect.  That is, for
some error threshold $\eps$, bag $B$ should contain $b(1\pm\eps)$
percentage of the toys.

What should the number of toys $t$ be, in terms
of $a, b, \eps$? If you fix $t$, what should the probabilities $a, b$ be
to ensure $\eps$ concentration in bag $B$?

\subsection{Notes}
Here, if $a = b = \frac{1}{2}$, then the number of toys $t$ should be
approximately $\eps^{-2}$ (unproven).

Suppose $t = \eps^{-1}$. Then what should $b$ be?


\section{Relation to Sampling, and obtaining Effective Resistance in
$O(\eps^{-1})$ time.}
Suppose you have a given vertex $v$, and a cut with two parts: $A$ and
$B$.  Let $v$ be in part $A$, and let it have $a'$ neighbors in $A$ and
$b'$ neighbors in $B$.

Here, we model degree-preservation on $v$ as placing down $a'+b'$ edges,
either in $A$ or in $B$.

Now, if we do leverage score sampling and sample $E$ edges, the probability any edge appears
is equal to the (leverage-score)$*|E|/n$.  If we set $E = n/\eps$, then
it becomes $(leverage-score)*\eps^{-1}$.

For any vertex of a unit graph, any edge leading into it has leverage
score at least $\frac{1}{d}$, and thus for each vertex the sum of
leverage scores is at least $1$. Therefore, the expected
number of edges is at least $\frac{1}{\eps}$. 

Assuming the worst case, the number of edges is equal to
$\frac{1}{\eps}$.



\end{document}
