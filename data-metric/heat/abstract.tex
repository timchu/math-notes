In this note, we how heat on vertices of a graph cool when
one fixed vertex is a heat sink.
We propose using the long-term cooling on a graph with one vertex
as a heat-sink, as a measure
of that vertex's centrality in the graph.
Intuitively, a sink at a central graph
vertex may cause heat to rush
out of the graph faster than a sink at a non-central
vertex.

We can also compute a $k$-way
partitioning of the graph, by picking the $k$ vertices that drain
heat the fastest out of a graph. Let these $k$ vertices be
designated as exemlars, and let
each vertex in the graph be assigned to the
exemplar that absorbed most its heat.
This notion of dissipating heat quickly in a graph can be
extended to edges, where we can find which edges dissipate heat
quickly. 

The hope is that this notion admits nice theorems, rather than
just heuristical math results. Alternate measures of Vertex
Centrality (degree centrailty, Katz Centrality, Eigenvector
    centrailty, etc.) might just be more useful, but this is
another deuce in the pot. Note that Pagerank actually kind of
sucks on Geometric graphs, because any $d$-regular graph has the
same pagerank per vertex.

%% # In this note, we examine the use of heat dissipation as a
%% # centrality measure, combining both Euclidean distance and
%% # many-paths formulation. We suspect that the heat dissipation is
%% # once again locally bottlenecked, but due to lack of certainty we
%% # forge bravely onwards in our exploration.
