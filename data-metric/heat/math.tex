\section{Modeling Heat in a Graph  with Cold Sinks}
We seek to model heat in a graph with cold sinks,
as time goes to infinity. Suppose we have graph $G$ with cold
sink at vertex $p$.
The heat $h$ at time $t$ given starting heats $h_0$ (at all
    points excluidng $p$) would be:

\[e^{t L_{-p} D^{-1}} h_0,\]
where $L_{-p}$ denotes the Laplacian with the $p^{th}$ row
and column removed. Note that
\[h = D^{1/2} e^{t D^{-1/2}L_{-p}D^{-1/2}} D^{-1/2}
h_0. \]
It appears to Tim Chu as if the behavior of this vector under any
measure as $t \rightarrow
\infty$, is determined by the smallest eigenvalue of 

\[ D^{-1/2} L_{-p} D^{-1/2}. \]
This is because the heat in the long term should be
entirely governed by $e^{-\lambda_1 t}$, where $\lambda_1$ is the
smallest eigenvalue of the normalized Laplacian. 

This eigenvalue
should govern the long-term heat as
long as $D^{-1/2} h_0$ has some nontrivial portion of $v_1$ (the
eigenvector corresponding to the smallest eigenvalue of the
normalized Laplacian). 
The long-term
heat in the graph under most measures is smallest when
$\lambda_1$ is large.  
An identical problem: given a
random walk, what vertex has the smallest long-term probability
that a random walk avoids it?


