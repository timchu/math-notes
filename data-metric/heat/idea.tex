The idea: measure centrality of a vertex $v$ as follows: given a
uniform distribution of heat in a graph, how quickly does heat
dissipate into a hole?

At the beginning of the time process, this is locally
bottlenecked. However, as time goes on, the hope is that
eventually the more global properties of the graph come into
play. Alternateively, a question can be how quickly heat spreads
out of a single point.

Here, there are many measures for how quickly heat dissipates.
Given the timeseries of heat on the graph
\begin{enumerate}
\item What is the integral of the heat?
\item What is a back-loaded integral of the heat? (Or
    front-loaded)
\item How fast does the heat dissipate at the onset? (Entirely
    local)
\item  What time $t$ does the total heat on the graph undershoot
$\eps$? (total heat on the graph having some measure or another).
\begin{itemize}
\item What happens as $\eps \rightarrow \infty$?
\end{itemize}
\item How quickly does the graph lose half its total heat?
\end{enumerate}
To simulate the heat sink, we start with a stable distribution on
heat over all the vertices. Then, we perform a lazy random walk
with hole. In particular, suppose the adjacency matrix is $A$ (so
    the probability transition matrix is $AD^{-1}$).
However, suppose we have a sink at point $P$. That is, the
$P^{th}$ row and column of $A$ are all zero. Define this matrix as
$A_{-p}$. Now the transition matrix is $A_{-p}D^{-1}$.

Let's simulate a lazy random walk. Then the transition matrix for
one timestep is:
$\lim_{a \rightarrow 0} ((1-a)I + a(A_{-p}D^{-1}) )^a$, or 

\[ e^{-L_{-p}D^{-1}} = D^{1/2} e^{\overline{L}} D^{-1/2}. \]

Here, $\overline{L} = D^{-1/2} L D^{-1/2}$.
We now seek to count the total heat in the graph. Unfortunately,
most obvious measures ($l_1$, $l_\infty$, etc.) are too hard to
  count. Therefore, I'm going to take the easiest measure of heat
  vector $h$ that I can, which will be:

  \[ h^T D^{-1} h. \]

Now the heat value becomes:

\[ \textbf{1} e^{-2L_{-p}t} \textbf{1}. \]

What we would like to do, is have an ordering on vertices $p$ as
$t \rightarrow \infty$ for when this value is smallest as $t$
gets high. That is, the behavior of the heat in the graph as time
grows large. Perhaps this is also not that interesting though. I
think getting a half life might be more interesting than this.

Now the question is: does the value of this number, in the limit,
depend mostly on the degree of $d$, or is it dependent on
other global factors of the graph?

As $t$ gets large, this should depend exclusively on the largest
eigenvalue of $D^{-1/2}L_{-p}D^{-1/2}$.

