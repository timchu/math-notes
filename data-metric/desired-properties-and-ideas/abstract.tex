In this paper, we list out a number of proposals for metrics on
graphs. We also sketch out impossibility results. By default,
edge weights are conductances and not resistances. The properties
we would like to have are:

\begin{enumerate}
\item A metric ball is connected. This is of use in machine
learning, when the underlying graph is local (or at least
semi-local).
\item If our graph is a line: $A-x$, $x-B$, then $A$ and $B$
should have some distance between them. If $x$ is connected to a
large graph, but $A$ is connected only to $x$ and $B$ only to
$x$, then either $A$ and $B$ get closer, or $A$ and $B$ stay the
same distance.
\item Rayleigh monotonicity property.
\item A model that has some kind of physical or simple
mathematical meaning. We speculate such a model has more
interesting mathematics underneath.
\end{enumerate}

We examine the use of htiting times, commute times, voltages,
$p$-voltages, maximum flow, low conductance cuts, shortest paths, and more. The
number of ways we can approach this problem is manifold:
\begin{enumerate}
\item Determine when a vertex is central.
\item Determine when a point $x$ is closer to $A$ than to $B$.
\item Build balls to fixed point $A$, and then create a
symmetrized measure from there. (Dijkstra, doing all pairs
    shortest path at once).
\item Build a measure that is some function of its neighbors.
\item Make a differential equation that dissipates things faster
than heat (like if there were positively charged protons at
    every vertex or something).
\end{enumerate}
It seems like I would need an alternative strategy. Level sets
and low conductance cuts don't quite fit into this mold.

It seems as if I would need to be very creative, or have
inspiration from some other sources, (crystallography,
    glass-making, gravitation, strong and weak forces, Maxwell,
    fluid mechanics, terrestrial mechanics, microbial action,
    neurons, humanities, or more) because many machine learners
have undoubtedly looked at this problem. Thus far, I have also
been -- unnecessarily -- negative on things like spectral
clusering or diffusion maps, and it seems worthwhile to look into
that line of work and their limitations.
Here's a few approaches that Tim Chu has tried, doesn't believe
them to work, and will list out why they don't work:
\begin{enumerate}
\item Heat half lives
\item Voltage level sets
\item Time weighted heat flow.
\item New symmetrization of heats.
\item Laplacian powers for Effective Resistance. (?)
\item Allowing the net flow at the start and end point to be distributed, at
some cost.
\item ????
\end{enumerate}
Some questions:
\begin{enumerate}
\item It would seem as if commute times should fail the $A-x$
$x-B$ test but they don't.
\end{enumerate}
 
%What I'm really interested in, is level sets based on voltage.
%Can a level set based on voltages give any insight about the
%graph? This is a function $f : G, V(G), V(G) \Rightarrow
%P(P(\R)}$. Or: take a graph, then two vertices, and generate a
%collection of nested sets.
%
%Gary suggests looking at the measure: testing whether $x$ is
%closer to $A$ or $B$, by determining whether the random walk hits
%$A$ or $B$ first. Intuitively, this should be entirely local.
%However, I haven't worked out the mathematics here.
%
%Does conductance cut determine how far apart two items are? Am I
%cutting the resistance or conductance? Does electrical flow in a
%solid work the same way as if a geometric graph was laid down on
%the the surface? Honestly probably not.
