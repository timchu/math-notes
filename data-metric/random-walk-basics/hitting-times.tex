
In this note, we examine diffusion. We cover:

(a) Hitting Times.

(b) Commute Times.

(c) Diffusion, and thoughts on how to study that phenomenon in greater
depth. This includes Gary's idea about the time it takes for the
Gaussian heat kernel to hit the inflection point. Ideally, we may
provide intuition for why hitting time is the same in the diffusive and
in the standard model.

This follows a line of reasoning that: diffusion, or its
generalizations, may be an interesting
property to study within graph networks. 
\section{Hitting Times}
In graph $G$, let $\huv$ denote the hitting time from $x$ to $y$. For
now, fix $u=1$, and shorthand $h_v= h_{1v}$. $h_v$ satisfies:

\begin{align}
  h_v = 1 + \sum_{x \in N(v)} \frac{c_{vx}}{c_v} h_x 
\end{align}
or:
\begin{align}
  c_v h_v = c_v + \sum_{x \in N(v)} c_{vx} h_x.
\end{align}
This holds for all $v \not= (u=1)$. Boundary condition: $h_u = 0$.

Then: this equation is equal to:

\begin{align}
  D \cdot h = \begin{pmatrix} c_1-C \\ c_2 \\ \ldots \\ c_n   \end{pmatrix}
    + A \cdot h
\end{align}
or:
\begin{align}
  (D - A)h = \begin{pmatrix}c_1 - C \\ c_2 \\ \ldots \\ c_n
  \end{pmatrix}
\end{align}
The $c_1-C$ term comes from: $D-A$ has image orthogonal to the all ones
vector. $c_1-C$ is the unique possible entry in that vector. I do not
know how to deduce $c_1-C$ term otherwise. 
\begin{align}
  h = L^\dag \begin{pmatrix} c_1 - C \\ \ldots \\ c_n \end{pmatrix}
\end{align}
plus a constant times the all ones vector. What the constant is, I do
not know. (It is to ensure boundary condition is right.)
\section{Commute Times}
$C_{xy}$ is commute time. $=h_{xy} + h_{yx} = $ (when $x = 1, y = n$):

\begin{align}
  &
  \begin{pmatrix} -1 , 0 , 0 , \ldots , 1\end{pmatrix} L^\dag
    \begin{pmatrix} c_1 - C \\ \ldots\\ c_n \end{pmatrix}
      \\
  &+
  \begin{pmatrix} c_1 , 0, 0 , \ldots , c_n-C\end{pmatrix} L^\dag
    \begin{pmatrix} 1 & 0 & 0 & \ldots -1 \end{pmatrix}
\end{align}

\begin{align}
  &
  = \begin{pmatrix} -1 , 0 , 0 , \ldots , 1\end{pmatrix} L^\dag
    \begin{pmatrix} c_1 - C \\ \ldots\\ c_n \end{pmatrix}
      \\
  &+
  \begin{pmatrix} -1 , 0 , 0 , \ldots 1 \end{pmatrix} L^\dag
    \begin{pmatrix} -c_1 &  & 0 & \ldots & C-c_n\end{pmatrix}
\end{align}
\begin{align}
  &
  = \begin{pmatrix} -1 , 0 , 0 , \ldots , -1\end{pmatrix} L^\dag
    \begin{pmatrix} - C \\ \ldots\\ C \end{pmatrix}
  \\
  &= C \cdot \chi_{1n} L^\dag \chi_{1n}
\end{align}
