\noindent Proposal: Random walks with dropoffs.

\noindent Input: Graph $G$, starting point $s$, ending point $t$,
  value $c(G)$.

\noindent Schema: Walk from $s$. At each timestep and vertex (for some $x$):
\begin{enumerate}
\item Quit-or-walk with probability $1$.
\item Walk with probability $d\cdot c$.
\item Quit with probability $1-d\cdot c$.
\end{enumerate}
The number of times that an agent following this random walk is
expected to be at point $t$ is:
\[ \sum_{t = 0}^\infty \chi_t^T(c \cdot A)^t \chi_s \]
or:
\[ \chi_t^T (I - cA)^{-1} \chi_s. \]
This is since the transition matrix is:
$v \rightarrow c \cdot Av$.
Assuming constant $c$, (so $c$ large compared to say, $d_max$), this
has Rayleigh property. However, if $c$ is dependent on $d_max$
then it's not so clear. Allegedly, this number increases with the
number of edges (so similarity increases).

\section{Further Directions}
\subsection{Probability the Random Walk Ends}
The probability a walk from $t$ hits $s$ at all is: (unproven):
\[ \frac{\chi_t^T (I - cA)^{-1} \chi_s}
{\chi_s^T (I - cA)^{-1} \chi_s}. \]
\subsection{Similarity to Distance}
Now define the metric: 
$(\chi_i - \chi_j)^T (I-cA)^{-1} (\chi_i - \chi_j)^T$. I would
love this to be decreasing in $A$, because distances should
follow Rayleigh. However, this is not the case. If I have a long
thin line starting from $s$ and ending at $t$, adding two thick
edges to the ends will drastically increases $\chi_s^T (I - cA)^{-1}
\chi_s$ but not $\chi_s^T (I-cA)^{-1} \chi_t$.

We may also consider effective resistance with the matrix:
\[ (I - cA)^{-1} \]
which bears similarity to the Laplacian. I suspect that the
effective resistance using this matrix will be connected. I
suspect this will also have the Rayleigh property, as adding an
edge to $A$ (without increasing $c$) will INCREASE (in Loewner
ordering) the matrix $(I -cA)^{-1}$. (what.) Since this is a rank
$2$ update.

%Let's calculate the probability the random walk finishes. I would
%like to do this without striking a row and column....
%
%
%Note: in the following, we are computing the expected number of times you hit the
%target in a random walk. Which... is not exactly the same as the
%probability the walk finishes....
%
%The transition matrix is:
%$v \rightarrow c \cdot Av$. Probability the walk finishes is:
%\[ \sum_{t = 0}^\infty \chi_t(c \cdot A)^t \chi_s \]
%or:
%\[ \chi_t (I - cA)^{-1} \chi_s. \]
%Here, we assume that $c < \frac{1}{d}$ so the inverse always
%exists. This should have Rayleigh property (adding an edge is the
%    same as rerouting a quitter).
%Balls
%should also be connected (the probability you finish is strictly
%    less than the probability you finish at the time step plus
%    $1$).
%
%This should have a lot of problems: $c$ needs to be chosen in
%advance. (This measure happens to be symmetric). The denser you
%are the better chance you have of finishing.
