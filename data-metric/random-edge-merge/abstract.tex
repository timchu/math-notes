Given a graph, we merge edge $e$ with probability proportional to
$f(w_e)$. Then we see if $a$ and $b$ are connected. The idea is
that each edge represents some probability they are in the same
cluser. (Maybe clustering is not the right idea.)

This clearly captures Rayleigh, but it is not clear if it also
runs into local bottlenecking problems.

If the graph is unit, then suppose each edge is red with
probability $p$. Let $p'$ be the value such that $a$ and $b$
are connected by red edges with probability $> 1/2$,
if each edge is red with probability $p'$. Then we can say $a$
and $b$ are close if $p'$ is small, and $a$ and $b$ are far if
$p'$ is big.

Unfortunately I don't know what this value is on a grid, even if
we restrict to up-right paths. Maybe this is the starting point
for an exploration? But this seems to deal much more with
percolation theory.
