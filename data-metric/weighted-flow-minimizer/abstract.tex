Let $G$ be a graph with doubling dimension $D$. Fix any vetex
  pair
$A$ and $B$. For each edge $p$, compute $\w_p = \min\{ d(p,A), d(p, B)
  \}^{d-1}w_p$. Now consider the flow minimizer:

\[ \sum_e \w_e f_e^2.\]
Technically, any of:
\[ \sum_e \w_e^{-(k-1)} f_e^k\]
or:
\[ \sum_e (\w_e f_e^2)^s\]
could have been chosen. But our choice of flow minimizer allows us to compute the
flows via effective resistance computation. These variables were
chosen so that if there were $r^{d-1}$ points away from point
$A$, then the total contribution of these points to the flow
would be small.

Ah, this measure is rayleigh, since adding an edge in the graph decreases
$\min \{d(p,A), d(p, B)\}$. Which decreases $w_p$. Which
decreases the flow minimizer. This assumes $d$ is fixed. 

Suppose we wanted to count the number of points approximately $r$
away in Euclidean distance. I think if you're willing to tolerate an $\eps$ error, you
can do this as quickly as building an $\eps$ WSPD. But if we want
to calculate the distance in the geodesic, then suddenly things
are more complicated....

\subsection{Computing using effective resistance flow?}
There's no way this will work, but it could be an interesting
crappy heuristic.  But computing this way is probably not
Rayleigh.

\subsection{Are balls connected?}
It's unclear. I would doubt it. Looks like it's not for the $m$
bond. Or is it? For the 5 step $m$ bond (3 vertices between the
two sides of the bond), the measure has value:

\[ \frac{1}{m}+\frac{2^{d-1}}{m} + \frac{1}{m} \]

If we found the distance from the middle vertex in the bond to
any of the sides, the value would be:

\[ 1 + 1\].

In general, the $m$ bond would have value:
\[ \frac{1}{m}+\frac{2^{d-1}}{m} + \ldots +  k^{d-1}{m}  + \ldots
+ \frac{1}{m} \]

Note that the doubling dimension of the $m$ bond is roughly
$\log_2 m$.

Meanwhile the path length is $k/2$.



connected.
\subsection{Computation Complexity}
Pre-compute all pairs shortest distance, for an overhead of
$O(n^2)$ with some huge $\eps$ dependencies.

Now for each pair, build the local graph. If the graph is size $E$
(probably linear in $n$ for low dimension), then the time it
takes to compute a single pair effective resistance is $O(E)$.
Then the total runtime is $O(n^2 E)$. Which is $O(n^3)$ plus some
huge exponential factors in dimension.

THis is pretty bad, even if we assume low dimension! What would
be great is if we could get down to quadratic time. 

\subsection{An alternate measure from geometry}
Geometric graphs are very well structured, so we might as well
compute some kind of Katz centrality to each vertex. How do we
find how close two points are? We can take the geodesic distance,
and multiply by the Katz centrality of both points. This takes
into account both absolute geodesic distance (which can be
    ridiculously shortcut) and Katz Centrality.



One great thing that I'd love, but seems hard to come by, is to
build a resistance sparsifier for one graph, and show that it
actually works great for all graphs. That sounds highly false
though, so I'm probably not gonna run it....


Perhaps the best I can hope for in this distance, is to compute a
resistance sparsifier for each pair of vertices, which would
still be too much space to store. Damn!
%% Previous ideas and mistakes:

% Oops, this measure is not Rayleigh. Adding an edge can decrease
% $\min \{d(p,A), d(p,B) \}$. Which increases
% $w_p$. But maybe also increases doubling dimension, which could
% save us. But if not, uh, maybe not the most useful for graphs that don't
% come from a distribution.
% 
% Note that this measure satisfies Rayleigh: adding an edge only
% decreases $\w_p$ for any $p$, and every flow minimizer is also
% Rayleigh.

% Let $A$ be a vertex, and measure its centrality via: computing
% the number of vertices in surface shell of radius $r$, and
% dividing by $r^{d-1}$.
% 
% Or: we can take an underlying graph, and then say: for any vertex
% $v$, weight it equal to the shortest path to $v$, times
% $1/k^{d-1}$. The reason the inverse weight is used is because
% this is normalized so that all vertices $k^{d-1]$ away on a graph 



% $A$ and $B$ in a graph are considered close if there are many
% paths from $A$ to $B$. Usually, this is captured by the idea of
% electric flow. However, electric flow suffers from the feature
% that (on geometric graphs) it is locally bottlenecked.
% That is,
% almost all the effective resistance from $A$ to $B$ comes from
% local neighborhoods around $A$ and $B$.
% 
% In this paper, we consider a measure on a flow from $A$ to $B$,
% weighted by shortest path. That is, if we suspect our graph
% has dimension $d$, then we weight the edge with the shortest path
% to that edge from either of the endpoints, weighted by dimension.
% 
% This captures the following intuition: If there are $k$ parallel
% paths from $A$ to $B$, then the....
% 
% Unfortunately, this is now dimensionally sensitive, which is not
% ideal for general graphs that have no strong notion of doubling
% dimension.
