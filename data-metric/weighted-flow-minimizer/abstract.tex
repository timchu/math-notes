$A$ and $B$ in a graph are considered close if there are many
paths from $A$ to $B$. Usually, this is captured by the idea of
electric flow. However, electric flow suffers from the feature
that (on geometric graphs) it is locally bottlenecked.
That is,
almost all the effective resistance from $A$ to $B$ comes from
local neighborhoods around $A$ and $B$.

In this paper, we consider a measure on a flow from $A$ to $B$,
weighted by shortest path. That is, if we suspect our graph
has dimension $d$, then we weight the edge with the shortest path
to that edge from either of the endpoints, weighted by dimension.

This captures the following intuition: If there are $k$ parallel
paths from $A$ to $B$, then the....

Unfortunately, this is now dimensionally sensitive, which is not
ideal for general graphs that have no strong notion of doubling
dimension.
