Intuitively, suppose we have some kind of manifold, surface, or
graph. Suppose I have two points $A$ and $B$. I would like some
notion of how well connected $A$ and $B$ are, with the following
caveats:

Becuase this method should work for geometric shapes, there must
be some idea of asking how many paths with width $w$, where $w$
scales with the distance from any point, there are between $A$
and $B$. In particular, you probably want to take some kind of
weighted integral between $A$ and $B$, of the volume of some
cross-section. Here, the weight should either scale with the
distance to either $A$ or $B$, or with respect to the volume of the
ball covered so far. 

That seems like a reasonable definition for a distribution. Here,
     we implicitly assume that the underlying Euclidean distance
     must matter.

Therefore, we simply integrate:

\[
  \int_{0}^{d(A, B)/2} p(r)^{-1} dr
\]
where $d$ is some distance (in this case the Euclidean distance),
and $p(r)$ is the integral of the probability denstiy function
along the $r$ radius ball.
\tim{There are a ton of problems with this: a discontinuous
  distribution can still have non-infinite measure here.}

Try number $2$: lets use the geodesic distance $d$, to measure
the lengtho f a path. Note that this doesn't measure connectivity
between two points, but it seems like a (moderately reasonable)
  measure of distance.

Then we take:

\[ \int_0^{d(A,B)....\]



An alternative is:  do you want to measure
Now the big question is:



$A$ and $B$ in a graph are considered close if there are many
paths from $A$ to $B$. Usually, this is captured by the idea of
electric flow. However, electric flow suffers from the feature
that (on geometric graphs) it is locally bottlenecked.
That is,
almost all the effective resistance from $A$ to $B$ comes from
local neighborhoods around $A$ and $B$.

There are two solutions to this: first, we can try and augment
the graph so that Effective Resistance is not locally
bottlenecked. I haven't explored this issue in depth, and perhaps
it is promising.

The second issue, the topic of this 
