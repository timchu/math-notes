\section{Diffusion Equation}
The diffusion equation: $e^{-LD^{-1}t}$.

\begin{align}
= D^{1/2} e^{-D^{-1/2}L D^{-1/2} t} D^{-1/2}
\end{align}

So: If you want to find the discrete expected hitting time (from vertex
    $1$), you would
solve:

\begin{align}
h_y = 1 + \sum_{x \in N_{df}(y)} \textbf{df}_{xy} h_x
\end{align}
where $\textbf{df}_x$ is the diffisuion matrix's value at $xy$ for $y \not= 1$.

or:

\[(I - {df})h = 1\]

or:

\[(I - D^{1/2} e^{-\tilde{L}t} D^{-1/2}) h = 1\]

or: 

\begin{align}
D - (D^{1/2} e^{-\tilde{L}t} D^{1/2}) h =
  \begin{pmatrix} c_1 - C \\ c_2
  \\ \ldots \\ c_n  \end{pmatrix}
\end{align}
Here, $c$ is the degree sequence, and $C$ is the total sum of the
degrees (or $2m$).

Commute time is:
\begin{align}
2m \chi_{xy} (D - (D^{1/2} e^{-\tilde{L}t} D^{1/2}) h) \chi_{xy}
\end{align}

I moved a little fast here, so confusions with my math may be errors
with my math. Here, $\tilde{L} := D^{-1/2}LD^{-1/2}$.

Thus, it may be of interest to examine the eigenvectors of the
discretized-time, continuous-walk matrix, or:

\begin{align}
D - D^{1/2} e^{-\tilde{L}} D^{1/2}
\end{align}

However, this seems to always be a Laplacian: $I - df$ should have
positive diagonals, as the random walk prescribed by $df$ doesn't stay in the same place with more
than probability $1$. Right mulltiplying by $D$ doesn't change the
positivity of diagonals. That is ,we're just replacing one transition
matrix (A) for another (the lazy random walk).

The above matrix also has all ones as eigenvector. So no new class of
matrices is generated in this fashion. But maybe this matrix is still
useful for finding distances between two points in a distribution....
