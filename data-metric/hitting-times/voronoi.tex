\section{Voronoi diagrams with Hitting Times}
Let $A_1, A_2, \ldots A_k$ be a collection of Voronoi sites, and
suppose we want to build a Voronoi diagram with respect to $f$.
In this section, we claim that:

The Voronoi diagram generated by $f$ (for any
function $f$ respecting the hitting time ordering) can be
generated through $H$ by assigning point $p$ to site:
\[ \argmin_i H(A_i, p). \]

\begin{theorem} Any Voronoi diagram generated through any $f$
respecting the hitting time ordering, must be the Voronoi diagram
listed above.
\end{theorem}
\begin{proof} You would want point $p$ to b assigned to site $\argmin_i f(A_i, p)$,
  which is the same as $\argmin_i H(A_i, p)$ by definition of
  $f$.
\end{proof}
Thus, our Voronoi cells are well defined.

\begin{theorem}\label{thm:connected} If there are only two
Voronoi sites, then the Voronoi cells are connected.
\end{theorem}
Theorem~\ref{thm:connected} has counterexamples when there are
three site. \tim{Draw it in}.

\begin{lemma} \label{lem:H}
\[ H(A, p) - H(B, p) = \frac{1}{d_p} \sum_{q \in N(p)} H(A, q) - H(B, q) \]
for all $p \not= A, B$.
\end{lemma}

\begin{proof}\textit{(of Theorem~\ref{thm:connected}, using
    Lemma~\ref{lem:H})}

Suppose there is a set of vertices disconnected from $A_1$, that
are closer to $A_1$ than any other vertex. Lemma~\ref{lem:H} tells
us that $H(A_1,p)-H(A_2,p) < 0$, so take the minimum $p$ on this
set of disconnected vertices. Since it's minimal, all its
neighbors must have the same value of that difference, and all of
those neighbors must too, and so forth. However, since the graph
is connected, this is impossible since the value of this
difference must be positive at $A_2$.
\end{proof}
\begin{proof}\textit{(of Lemma~\ref{lem:H})}

Recall 
\[ H(A, p) - H(B, p) = (\chi_A - \chi_B)^T L^\dag (\overline{d} - 2m
    \chi_p) \]
\[ H(A, q) - H(B, q) = (\chi_A - \chi_B)^T L^\dag (\overline{d} - 2m
    \chi_q) \]

So 
\begin{align}
\frac{1}{d_p} \sum_{q \in N(p)} H(A, q) - H(B, q)
= (\chi_A - \chi_B)^T L^\dag (\overline{d} -2m
\sum_{q \in N(p)} \frac{1}{d_p} \chi_q).
\label{eq:main}
\end{align}

Now, 
\[  \frac{1}{d_p} \sum_{q \in N(p)} \chi_q 
= AD^{-1} \chi_p. \]
Therefore,  we can simplify Equation~\ref{eq:main} to:
\begin{align}
& \frac{1}{d_p} \sum_{q \in N(p)} H(A, q) - H(B, q)
  \\
& = (\chi_A - \chi_B)^T L^\dag (\overline{d} -2m
\sum_{q \in N(p)} \frac{1}{d_p} \chi_q).
\\
&= (\chi_A - \chi_B)^T L^\dag \overline{d} -2m(\chi_A -
    \chi_B)^T (L^\dag AD^{-1} \chi_p) \label{eq:second}
\end{align}
However, note
\[L^\dag (D-A) D^{-1} = I_{\bot \textbf{1}} D^{-1}, \]
or 
\[ L^\dag A D^{-1} = L^\dag - I_{\bot \textbf{1}} D^{-1},\]
so Equation~\ref{eq:second} equals
\[ (\chi_A - \chi_B)^T L^\dag \overline{d} -2m(\chi_A -
    \chi_B)^T (L^\dag - I_{\bot \textbf{1}} D^{-1}) \chi_p \]
or
\[ (\chi_A - \chi_B)^T L^\dag \overline{d} -2m(\chi_A -
    \chi_B)^T L^\dag \chi_p  +2m (\chi_A - \chi_B)^T I_{\bot \textbf{1}} D^{-1}) \chi_p \]
    but note that the last term in the summand is $0$, since
    $(\chi_A - \chi_B)^T I_{\bot \textbf{1}}  = (\chi_a -
        \chi_B)^T$, which is orthogonal to $D^{-1} \chi_p$, so 

\[ \frac{1}{d_p} \sum_{q \in N(p)} H(A,q)-H(B,q) = H(A, p) - H(B,
    p) \]
as desired.
\end{proof}
