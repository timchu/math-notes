Expected hitting times are widely used measures in combinatorics and data
mining~\cite{Chen08}. The work of Von Luxburg et.
al.~\cite{VL14} showed that hitting times are largely meaningless
measures on large-scale, random geometric graphs such as the
$k$-NN graphs, $\eps$-graphs, and Gaussian similarity graphs.
They proved that hitting time from $a$ to
$b$ captures only local information of a
graph, as it converges to $\frac{1}{d_b}$.

Since hitting times are asymmetric, often it can be more useful
to deal with a symmetric measure.
Commute times are the most popular symmetrization of hitting
times, defined as the expected time to travel from vertex $a$ to
$b$ and back again. Commute times are known to be scaled
versions of effective resistance, and also capture few global
properties of a random geometric graph: commute times converge to
$\frac{1}{d_b} + \frac{1}{d_a}$ in $\eps$-graphs and $k$-NN
graphs~\cite{VL14}. Additionally, balls of radius $r$ measured in
commute time are not necessarily connected, which makes commute
times an unsuitable similarity measure in some machine learning
settings.

The key result of this paper is to generate a new symmetrization
of
hitting times that captures the global properties of the
graph, avoiding Von Luxburg's problem. 
Namely, we build a symmetric
function $f$ such that $f(a,b) > f(c,b)$ if and only if the
hitting time from $a$ to $b$ is larger than the hitting time from
$c$ to $b$. This way, $f$ captures the \textit{partial ordering}
of hitting times given a fixed destination, rather than concern
itself with the actual value of the hitting time. In this paper,
       we show:
\begin{enumerate}
\item Function $f$ exists,
\item $f$ avoids von Luxburg's general problem with hitting
time, and computes global properties of underlying graph $G$
\item Balls centered at point $p$ with radius
$r$
(where this ball is defined as $\{x \in V(G): f(x, p) \leq r\}$) are connected, and
\item  The Voronoi cells of $f$ are the same for any
function $f$ respecting the partial ordering of hitting times,
         and show that when there are two Voronoi sites then each
         Voronoi cell is connected.
\end{enumerate}
The last property may make our symmetrized hitting time $f$
suitable for bisecting $k$-means, as Voronoi cell computation is
an important primitive in $k$-means clustering. These Voronoi cells can be
computed without explicit access to $f$.  
This symmetric hitting time $f$
captures the additional desirable feature that two points $A$ and
$B$ are considered closer if there are many paths from $A$ to
$B$.

\tim{More things: Explicitly construct $f$ rather than
  implicitly}
\tim{Maybe explicit $f$ will be nice}
\tim{Random walks with resets? Other kind sof random walks?}
\tim{Half live?}
\subsection{Preliminaries}
\begin{definition} The hitting time $H_G(a,b)$ from vertex $a$ to
$b$ in graph $G$ is the expected time it takes for a random walk
from $a$ to hit $b$. It can be alternatively defined as follows:

\[ H_G(a,b) \defeq (\chi_a - \chi_b)^T L_G^\dag (\overline{d} -
    \chi_b).\]

Here, $\chi_a$ is the indicator vector for vertex $a$, and
$\overline{d} \in \mathbb{R}^{|V(G)|}$ is the
vector with $\overline{d}_i$ equal to the degree of vertex $i$.
\end{definition}
%% 
%% Hitting times have two desirable properties which we
%% seek to exploit in this paper: if $A$ and $C$ are
%% connected by many disjoint paths of This symmetrization
%% connected by only one path of length $l$, then $H(C, A) \leq
%% H(B, A)$. Moreover, the set of vertices $v$ such that $H(v, A)
%% \leq t$ for any threshold $t$ is connected. This
%% means any ball (radius measured using hitting times to $A$) is
%% connected and contains $A$. We use these properties as the
%% starting point in generating our symmetrization of hitting
%% times.
