\section{Problem with this approach}
No rayleigh.
%% The hitting time metric I described does not have Rayleigh
%% property, which seems very important. If $A$ and $B$ are
%% connected by many paths, and $A$ and $C$ are not, but then I
%% decide to hang a ton of other paths onto $B$ leading away from
%% $A$, then $A$ and $B$
%% become less connected.
%% 
%% Normalizing by degree doesn't necessarily help (or does it?).
%% Does this concept of hitting times work for $d$-regular graphs?
%% 
%% Maybe this works fine for $d$ regular graphs. That is, hitting
%% times to a destination may measure how many paths there are to
%% that destination. But even then, I think I can build
%% counterexamples. For instance, if I hang one paths onto $B$ that
%% lead to big chunks of the graph far from $A$, that's makes $B$
%% far from $A$. Meanwhile, if I hung one path leading to a small
%% tightly connected cluster away from $A$, that makes $B$ closer to
%% $A$. Is that a bad property, or maybe that's good?
%% 
%% But for non $d$-regular graphs, I'm at a loss. It seems like I'll
%% want to do something other than hitting time, or even
%% degree-normalized hitting time. 
