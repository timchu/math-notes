\documentclass[12pt]{article}
\usepackage{fullpage}

\usepackage{amsmath}
\usepackage{amssymb}
\usepackage{amsthm, algorithm}
\usepackage{hyperref}
\usepackage{color}

\newtheorem{problem}{Problem}
\newtheorem{theorem}{Theorem}[section]
\newtheorem{prop}[theorem]{Proposition}
\newtheorem{corollary}{Corollary}[theorem]
\newtheorem{remark}{Remark}[theorem]
\newtheorem{lemma}[theorem]{Lemma}
\newtheorem{definition}[theorem]{Definition}
\newtheorem{observation}[theorem]{Observation}
\newtheorem{hole}{Hole}[theorem]

\newcommand{\cupdot}{\mathbin{\mathaccent\cdot\cup}}

\newcommand{\tr}{\mbox{Trace}}
\newcommand\prob[2]{\mbox{Pr}_{#1}\left[ #2 \right]}
\newcommand\expec[2]{{\mathbb{E}}_{#1}\left[ #2 \right]}
\newcommand\var[2]{\mbox{\bf Var}_{#1}\left[ #2 \right]}


\newcommand\Ctil{\widetilde{\mathit{C}}}
\newcommand\Otil{\widetilde{\mathit{O}}}

\newcommand\Ccal{\mathcal{C}}
\newcommand\Hcal{\mathcal{H}}

\renewcommand\AA{\boldsymbol{\mathit{A}}}
\newcommand\DD{\boldsymbol{\mathit{D}}}
\newcommand\MM{\boldsymbol{\mathit{M}}}
\newcommand\MMcal{\boldsymbol{\mathcal{M}}}
\newcommand\MMbar{\boldsymbol{\overline{\mathit{M}}}}
\newcommand\MMhat{\boldsymbol{\widehat{\mathit{M}}}}
\newcommand\II{\boldsymbol{\mathit{I}}}
\newcommand\LL{\boldsymbol{\mathit{L}}}
\newcommand\LLtil{\widetilde{\boldsymbol{L}}}
\newcommand\PP{\boldsymbol{\mathit{P}}}
\newcommand\UU{\boldsymbol{\mathit{U}}}
\newcommand\XX{\boldsymbol{\mathit{X}}}
\newcommand\XXcal{\boldsymbol{\mathcal{X}}}
\newcommand\XXJcal{\boldsymbol{\mathcal{X}}\mathcal{J}}
%\newcommand\YY{\boldsymbol{\mathit{Y}}}
%\newcommand\YYcal{\mathcal{Y}}
\newcommand\ZZ{\boldsymbol{\mathit{Z}}}
\newcommand\ZZhat{\boldsymbol{\widehat{\mathit{Z}}}}

\newcommand\cchi{\boldsymbol{\chi}}

\newcommand\simuniform{{\sim_{{\rm uniform}}}}

\newcommand\Ical{\mathcal{I}}

\newcommand\yhat{\widehat{y}}

\newcommand\AbsMatrix[1]{\mbox{Abs}_{2}\left| #1 \right|}


\newcommand\dd{\boldsymbol{\mathit{d}}}
\newcommand\rr{\boldsymbol{\mathit{r}}}
\newcommand\ww{\boldsymbol{\mathit{w}}}
\newcommand\xx{\boldsymbol{\mathit{x}}}
\newcommand\yy{\boldsymbol{\mathit{y}}}
\newcommand\eps{\varepsilon}


\newcommand\PPi{\boldsymbol{\mathit{\Pi}}}
\newcommand\one{\vec{1}}

\newcommand{\todo}[1]{{\bf \color{red} TODO: #1}}
\newcommand{\richard}[1]{{\bf \color{green} RICHARD: #1}}
\newcommand{\junxing}[1]{{\bf \color{green} JUNXING: #1}}
\newcommand{\saurabh}[1]{{\bf \color{green} SAURABH: #1}}
\newcommand{\sushant}[1]{{\bf \color{green} SUSHANT: #1}}
\newcommand{\tim}[1]{{\bf \color{red} TIMOTHY: #1}}


\begin{document}
\subsection{Sparsification matching Spielman bounds}
In this subsection, we describe a sparsification algorithm that involves flipping a fair coin on each edge, to determine wehther we keep it in the sparsifier. We iterate this $\log \frac{m}{n}$ times.

This is known \tim{ is it? }. Our algorithm for degree-preserving sparsification on unit weight graphs will use the same technique.

\subsubsection{Algorithm Description} \label{subsubsection:alg-desc}
\begin{enumerate} 
\item Take the graph $G$, and find all edges with effective resistance less than $2(n-1)/m$ (twice the average effective resistance).
\item Toss a fair coin on those each of those edges, to determine whether to either double
  their weight or remove them from the graph.
\item Iterate on the resulting graph for $O(\log \frac{m}{n})$ rounds.
\end{enumerate}
\subsubsection{Proving this algorithm works}
\begin{lemma} \label{lem:edge-keeps}
At least $\frac{1}{2}$ of the edges in the graph have effective resistance less than $\frac{2(n-1)}{m}$. \end{lemma}

\begin{proof} The average effective resistance of edges in a graph is $\frac{n-1}{m}$, via trace arguments. The rest follows from Markov's inequality. \end{proof}

  First, we state Matrix Bernstein, and a direct lemma of Matrix Bernstein \tim{which may or may not be Matrix Azuma.} 
  \begin{theorem} (Matrix Bernstein)
    Let $X = \sum_{i=1}^s S_i$. The norm of the total variance of $X$ is:
    \begin{align}
      \sigma^2 = \left| \left| \sum_{i=1}^s \expec{S_i} {S_i^2} \right| \right|
    \end{align}
    Then $\prob{}{||X|| \geq t} \leq d \cdot e^{\frac{-t^2}{\sigma^2 + Lt/3}}$
    where $d$ is the dimension of $X$ and $L = max||S_i||$. Here, $||M ||$ represents the spectral norm of $M$.

    (\tim{Check if d is defined the right way})
    %% \tim{Dimension of X? Can't X be low dimension even if the S are in high dimension?}
  \end{theorem}
  \begin{lemma} \label{lem:matrix-azuma}
    Let $X = \sum_{i=1}^s S_i$ and $S_i := \mu_i M_i$ where $M_i$ is a fixed PSd matrix and
    $\mu_i$ is a Rademacher random variable, 
    then $\prob{}{||X|| \geq t} \leq d \cdot e^{\frac{-t^2}{L \cdot \expec{S_i}{X}+ Lt/3}}$
    where $d$ is the dimension of $X$ and $L = max||S_i||$.
    Here, $||Y ||$ represents the spectral norm of $Y$.
  \end{lemma}
  \begin{proof} This follows from
    \begin{align}
    \expec{S_i} {S_i^2} \preceq \max_{i}||M_i|| \sum{M_i} 
    \end{align}
    when $M_i \succeq 0$.

    \tim{I'd much rather define a matrix absolute value and use $abs(S_i)$
    rather than $M_i$}
  \end{proof}

  \begin{lemma} Let $G'$ be the output of the algorithm stated in
  \ref{subsubsection:alg-desc}. Then $G'$ is with $\frac{1}{n^{O(1)}}$ probability an $1+O\left(\sqrt{\frac{n \log n}{m}}\right)$ sparsifier, with high probability. \end{lemma}
    \begin{proof}
      Consider the Laplacian $L_e$ corresponding to each edge of the graph $G$. Let
      $E'$ be the set of edges in $G$ with effective resistance less than
      $\frac{2(n-1)}{m}$.
      
      Let $S_e$ be the set of independent random variables such that: it is
      $L_e$ with probability $\frac{1}{2}$ and $-L_e$ with probability
      $\frac{1}{2}$. Recall $G'$ is the output of the algorithm in
      \ref{subsubsection:alg-desc}. Then 

      \begin{align}
        L_{G'} := \left(\sum_{e \in E} L_e \right) + \left(\sum_{e in E'}
      S_e\right).
      \end{align}

      by construction. If we left and right multiply this equation by $L_G^{\dag/2}$ and
      substitute $X_{H} := L_G^{\dag/2} L_{H} L_G^{\dag/2}$ for any graph $H$,
      and let $T_e := L_G^{\dag/2} S_{e} L_G^{\dag/2}$ then we
      get: 

      \begin{align}
      X_{G'} := \left(\sum_{e \in E} X_e\right) + \left(\sum_{e \in E'}
      T_e\right).
      \end{align}

      Note that $\expec{T_e}{X_{G'}} = I$ and $T_e$ are independent random
      variables with expected value zero satisfying (by construction):

      \begin{align}
      ||T_e|| \leq \frac{2(n-1)}{m}
      \end{align}
      and 


      \begin{align}
      \sum_{e \in E'} L_e \preceq I
      \end{align}

      then we can apply lemma \ref{lem:matrix-azuma} to get:

      \begin{align}
      \prob{}{||X_{G'} || \geq t} \leq n \cdot
      e^{\frac{-t^2}{\frac{2(n-1)}{m} + \frac{2(n-1)t}{m}}}
      \end{align}

      %% Note that the number of independent random variables we summed didn't
      %% actually matter one iota for matrix variance -- the important part is that
      %% T_e's sum \preceq I. 

      Note that if $t = O\left(\sqrt{\frac{n \log n}{m}}\right)$ (and $t \leq 1$), then 

      \begin{align}
      n \cdot e^{\frac{-t^2}{\frac{2(n-1)}{m} + \frac{2(n-1)t}{m}}}
      = \frac{1}{n^{O(1)}} 
      \end{align}


    thereby proving that 
      
      \begin{align}
      ||X_{G'} - I|| \leq O\left(\sqrt{\frac{n\log n}{m}}\right)
      \end{align}
        with polynomially high probability.
      which proves that $L_{G'}$ is a $1 + O\left(\sqrt{\frac{n\log n}{m}}\right)$
      sparsifier of $L_G$ with
      high probability.
    \end{proof}
    \tim{I will do a major cleanup on this part...}
    Some words about how chaining this lemma we just proved, with log n
    iteration, causes error to blow up geometrically each time, with factor
    $\sqrt{2}$ which is fine, 
    making the total error $\epsilon$ thanks to convergence of geometric series.

    Likewise, the run time of a single iteration of the algorithm described in
    \ref{subsubsection:alg-desc} is also nearly linear in $m$, and is thus
    dominated by the first iteration (as edges decreases by a $\frac{1}{2}$
    factor each time.

    \subsection{Obtaining a degree-preserving sparsifier of size 
    $O\left(\frac{nl \log n}{\eps^2}\right)$ in $X$ time, given an
    $(O(n), l)$ short-cycle decoposition}

    \tim{This is a very very rough shoddy writeup. The key part is that I
    believe you only pick up an $l$ factor in space.  Note that I think you can
    do better, in that I think you can pick up some sublinear function of $l$
    because I naively bounded the effective resistance of a cycle to be the sum
    of the effective resistances of the edges, and also did some matrix absolute
    value shenanigans under the hood when bounding the variance parameter. Both
    if these assumptions seem a little silly.}

    Cycles change nothing except the effective resistance calculation (and now
    we sparsify down to $\frac{3}{4}$ as many edges, in each step of te
    algorithm, which ultimately changes nothing).

    This throws in an extra $l$ factor into the variance squared,
    \tim{not an $l^2$ factor, which is what I initially expected}.

    Therefore, $t$ is now $= O\left(\sqrt{\frac{nl \log n}{m}}\right)$ (and $t
    \leq 1$), and so the final number of edges $m'$ we can sparsify to while
    incurring an $\epsilon^2$ error, satisfies:

    \begin{align}
       \eps = O\left(\sqrt{\frac{nl \log n}{m'}}\right)
    \end{align}
    or
    \begin{align}
       \eps^2 = O\left(\frac{nl \log n}{m'}\right)
    \end{align}
    or
    \begin{align}
       m' = O\left(\frac{nl \log n}{\eps^2}\right)
    \end{align}
    The runtime is more than linear in $m$, so since the number of edges
    decreases geometrically at each iteration of our coin-tossing algorithm,
    the total run time is a constant factor of the runtime of the first iteration.

    That is to say, the total run time is the time it takes to run a short cycle
    decomposition and estimate effective resistances on each cycle.

    \tim{I gave some thought into how degree preserving property might help us
    get a better sparsfier than the naive $l$ factor in space we pick up,
    where $l$ is the length of the short cycles in a short-cycle decomposition.
    I had no idea how to proceed, for a couple good reasons I will not go into here.}
\end{document}
   


