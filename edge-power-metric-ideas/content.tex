In CMS17, it is proven that the nearest neighbor metric equals
the edge squared metric. Formally:

\begin{definition}
\textbf{Edge $q$-power metrics} are defined as taking a point set
$P$, taking the Euclidean distance between points $l$, and
applying the function:

\[ f(l) := (l/2)^q/q \]

Then the shortest path on the resulting graph is computed.  This
shortest path is the edge $q$-power metric.
\end{definition}

\begin{definition}
\textbf{NN $q$-power metrics} wtih respect to point set $P$ are defined as taking the cost
function:

\[ d(x, P)^{q} \]

and integrating this over $x$ along a path.
\end{definition}

CMS'17 showed that the edge $q$-power metric and the NN $q-1$
power metric were equal for $q = 2$, and not true for $q < 2$
(even for just two points, with the exception of $q = 1$. For $q
 \leq 1$, the shortest path is always a straight line).

\section{Open Questions}
\begin{enumerate}
\item What are the approximation factors you get from using
CFMSV 15?
\item What are the approximation factors you get from naively
using CMS 17?
\item What are the approximation factors you can get using flows
on screw simplices? (Using a naive potential function).
\item Can you prove or disprove that you can use: certain restrictive classes of
potential functions that generate flows? (List three or so
    classes of potential functions to restrict our search.
    Really, there's an entire suite of such functions.)
\end{enumerate}
