\documentclass[12pt]{article}
\usepackage{fullpage}

\usepackage{amsmath}
\usepackage{amssymb}
\usepackage{amsthm, algorithm}
\usepackage{hyperref}
\usepackage{color}

\newcommand{\defeq}{\stackrel{\textup{def}}{=}}
\newtheorem{problem}{Problem}
\newtheorem{theorem}{Theorem}[section]
\newtheorem{prop}[theorem]{Proposition}
\newtheorem{corollary}{Corollary}[theorem]
\newtheorem{remark}{Remark}[theorem]
\newtheorem{lemma}[theorem]{Lemma}
\newtheorem{definition}[theorem]{Definition}
\newtheorem{observation}[theorem]{Observation}
\newtheorem{hole}{Hole}[theorem]

\newcommand{\cupdot}{\mathbin{\mathaccent\cdot\cup}}

\newcommand{\tr}{\mbox{Trace}}
\newcommand\prob[2]{\mbox{Pr}_{#1}\left[ #2 \right]}
\newcommand\expec[2]{{\mathbb{E}}_{#1}\left[ #2 \right]}
\newcommand\var[2]{\mbox{\bf Var}_{#1}\left[ #2 \right]}


\newcommand\Ctil{\widetilde{\mathit{C}}}
\newcommand\Otil{\widetilde{\mathit{O}}}

\newcommand\Ccal{\mathcal{C}}
\newcommand\Hcal{\mathcal{H}}

\renewcommand\AA{\boldsymbol{\mathit{A}}}
\newcommand\DD{\boldsymbol{\mathit{D}}}
\newcommand\MM{\boldsymbol{\mathit{M}}}
\newcommand\MMcal{\boldsymbol{\mathcal{M}}}
\newcommand\MMbar{\boldsymbol{\overline{\mathit{M}}}}
\newcommand\MMhat{\boldsymbol{\widehat{\mathit{M}}}}
\newcommand\II{\boldsymbol{\mathit{I}}}
\newcommand\LL{\boldsymbol{\mathit{L}}}
\newcommand\LLtil{\widetilde{\boldsymbol{L}}}
\newcommand\PP{\boldsymbol{\mathit{P}}}
\newcommand\UU{\boldsymbol{\mathit{U}}}
\newcommand\XX{\boldsymbol{\mathit{X}}}
\newcommand\XXcal{\boldsymbol{\mathcal{X}}}
\newcommand\XXJcal{\boldsymbol{\mathcal{X}}\mathcal{J}}
%\newcommand\YY{\boldsymbol{\mathit{Y}}}
%\newcommand\YYcal{\mathcal{Y}}
\newcommand\ZZ{\boldsymbol{\mathit{Z}}}
\newcommand\ZZhat{\boldsymbol{\widehat{\mathit{Z}}}}

\newcommand\cchi{\boldsymbol{\chi}}

\newcommand\simuniform{{\sim_{{\rm uniform}}}}

\newcommand\Ical{\mathcal{I}}

\newcommand\yhat{\widehat{y}}

\newcommand\AbsMatrix[1]{\mbox{Abs}_{2}\left| #1 \right|}


\newcommand\dd{\boldsymbol{\mathit{d}}}
\newcommand\rr{\boldsymbol{\mathit{r}}}
\newcommand\ww{\boldsymbol{\mathit{w}}}
\newcommand\xx{\boldsymbol{\mathit{x}}}
\newcommand\yy{\boldsymbol{\mathit{y}}}
\newcommand\eps{\varepsilon}


\newcommand\PPi{\boldsymbol{\mathit{\Pi}}}
\newcommand\one{\vec{1}}

\newcommand{\todo}[1]{{\bf \color{red} TODO: #1}}
\newcommand{\richard}[1]{{\bf \color{green} RICHARD: #1}}
\newcommand{\junxing}[1]{{\bf \color{green} JUNXING: #1}}
\newcommand{\saurabh}[1]{{\bf \color{green} SAURABH: #1}}
\newcommand{\sushant}[1]{{\bf \color{green} SUSHANT: #1}}
\newcommand{\tim}[1]{{\bf \color{red} TIMOTHY: #1}}

\begin{document}

\title{Title}
\author{
  Timothy Chu \\
  Carnegie Mellon University\\
  \texttt{tzchu@andrew.cmu.edu}
}

\setcounter{page}{0}
\maketitle
\thispagestyle{empty}
\begin{abstract}
Buser shows that the isoperimetric cut of a surface with Ricci
curvature bounded below, can be bounded as

$\lambda_1 \leq C(H + H^2)$

where $H$ is the isoperimetric ratio of a cut.

The first proof technique uses an "area minimizing current", which
shows that the curvature of the cut must be constant. 

The second proof technique doens't use this: instead, they use an
$\epsilon$ net to build a new division from the initial cut $X$ (which
can be 'wavy'). This new 'cut' (doesn't have to be a hypersurface,
isn't null, but can be full dimensional) is $\hat{X}$, and they do this by
defining $\hat{X}$ to be the locus of points where a ball of radius
$r$ has exactly half the mass in $A$ and half the mass in $B$. This
'smooths' wavy cuts.

Then they build an epsilon net around $\hat{X}$, and extend that to an
epsilon net of the whole manifold. Finally, they build a little collar
around $\hat{X}$, and use some properties of the hyperbolic spheres
(only useful since hyperbolic sphere is some manifold lower bound on
some volume terms relevant in the calculation, on objects of
lower-bounded Ricci curvature) to establish the final bound.

In short, they build a Rayleigh quotient using $\hat{X}$, a smoothed
version of $X$ (that isn't exactly a 'cut', but still divides manifold
$M$ into two pieces), via a collar of radius $r$ around $\hat{X}$. The
Rayleigh quotient is $1$ on the parts outside of $\hat{X}$, while it
is $d(p, \hat{X})/r$ everywhere else.

$\epsilon$-nets are used for the analysis, but I am not sure they are
used for the construction.
\end{abstract}

\clearpage
\bibliographystyle{alpha}
\bibliography{refs}
\begin{appendix}
\end{appendix}
\end{document}
