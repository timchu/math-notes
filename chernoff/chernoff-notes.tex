\documentclass[12pt]{article}
\usepackage{fullpage}

\usepackage{amsmath}
\usepackage{amssymb}
\usepackage{amsthm, algorithm}
\usepackage{hyperref}
\usepackage{color}

\newcommand{\defeq}{\stackrel{\textup{def}}{=}}
\newcommand{\argmin}{\text{arg min}}
\newtheorem{problem}{Problem}
\newtheorem{theorem}{Theorem}[section]
\newtheorem{prop}[theorem]{Proposition}
\newtheorem{corollary}{Corollary}[theorem]
\newtheorem{remark}{Remark}[theorem]
\newtheorem{lemma}[theorem]{Lemma}
\newtheorem{definition}[theorem]{Definition}
\newtheorem{observation}[theorem]{Observation}
\newtheorem{hole}{Hole}[theorem]

\newcommand{\cupdot}{\mathbin{\mathaccent\cdot\cup}}

\newcommand{\tr}{\mbox{Trace}}
\newcommand\prob[2]{\mbox{Pr}_{#1}\left[ #2 \right]}
\newcommand\expec[2]{{\mathbb{E}}_{#1}\left[ #2 \right]}
\newcommand\var[2]{\mbox{\bf Var}_{#1}\left[ #2 \right]}


\newcommand\Ctil{\widetilde{\mathit{C}}}
\newcommand\Otil{\widetilde{\mathit{O}}}

\newcommand\Ccal{\mathcal{C}}
\newcommand\Hcal{\mathcal{H}}

\renewcommand\AA{\boldsymbol{\mathit{A}}}
\newcommand\DD{\boldsymbol{\mathit{D}}}
\newcommand\MM{\boldsymbol{\mathit{M}}}
\newcommand\MMcal{\boldsymbol{\mathcal{M}}}
\newcommand\MMbar{\boldsymbol{\overline{\mathit{M}}}}
\newcommand\MMhat{\boldsymbol{\widehat{\mathit{M}}}}
\newcommand\II{\boldsymbol{\mathit{I}}}
\newcommand\LL{\boldsymbol{\mathit{L}}}
\newcommand\LLtil{\widetilde{\boldsymbol{L}}}
\newcommand\PP{\boldsymbol{\mathit{P}}}
\newcommand\UU{\boldsymbol{\mathit{U}}}
\newcommand\XX{\boldsymbol{\mathit{X}}}
\newcommand\XXcal{\boldsymbol{\mathcal{X}}}
\newcommand\XXJcal{\boldsymbol{\mathcal{X}}\mathcal{J}}
%\newcommand\YY{\boldsymbol{\mathit{Y}}}
%\newcommand\YYcal{\mathcal{Y}}
\newcommand\ZZ{\boldsymbol{\mathit{Z}}}
\newcommand\ZZhat{\boldsymbol{\widehat{\mathit{Z}}}}

\newcommand\cchi{\boldsymbol{\chi}}

\newcommand\simuniform{{\sim_{{\rm uniform}}}}

\newcommand\Ical{\mathcal{I}}

\newcommand\yhat{\widehat{y}}

\newcommand\AbsMatrix[1]{\mbox{Abs}_{2}\left| #1 \right|}


\newcommand\dd{\boldsymbol{\mathit{d}}}
\newcommand\rr{\boldsymbol{\mathit{r}}}
\newcommand\ww{\boldsymbol{\mathit{w}}}
\newcommand\xx{\boldsymbol{\mathit{x}}}
\newcommand\yy{\boldsymbol{\mathit{y}}}
\newcommand\eps{\varepsilon}


\newcommand\PPi{\boldsymbol{\mathit{\Pi}}}
\newcommand\one{\vec{1}}

\newcommand{\todo}[1]{{\bf \color{red} TODO: #1}}
\newcommand{\richard}[1]{{\bf \color{green} RICHARD: #1}}
\newcommand{\junxing}[1]{{\bf \color{green} JUNXING: #1}}
\newcommand{\saurabh}[1]{{\bf \color{green} SAURABH: #1}}
\newcommand{\sushant}[1]{{\bf \color{green} SUSHANT: #1}}
\newcommand{\tim}[1]{{\bf \color{red} TIMOTHY: #1}}

\begin{document}
\section{Chernoff via Bernstein:}
  If you have a bunch of variables $A_i$ that are $\leq L$ and with zero sum (0-1 coin flips are just a shifted and scaled version of this), then
  \begin{align}
  \prob{A_i \forall i}{\sum_{i=1}^s A_i \geq t}
  \end{align}
  with probability 
  \begin{align}
    \exp\left(\frac{t^2/2}{\var{A_i \forall i}{(\sum_{i=1}^s A_i)} + Lt/3}\right)
  \end{align}
  or alternately
  \begin{align}
    \max\left(\exp\left(\frac{t^2/2}{2Var(x)}\right), \exp\left(\frac{t^2/2}{Lt/3}\right)\right)
  \end{align}

  \section{Implications}
  \subsection{If you have a $\pm 1$ coin, and you want to figure out the number of flips you need such that you get less than $s/10$ from $0$ (the mean) with $1/2$ probability.}
  Here, of course, $s/10$ can be replaced with $s\epsilon$.
  Substitute:
  \begin{enumerate}
    \item $L = 1$
    \item $t = s/10$
    \item $var= s$ (variance of independent variables add).
  \end{enumerate}
  \tim{The following is wrong: we already know what t is..... here s must equal something like 100 or 50.}
  Then the condition of obtaining a bounded sum with at least $1/2$ probability is equivalent to showing that $t^2/2 = O(s)$, (from the variance term in th denominator of Bernstein, the otehr term is negligible [unproven!]).
  Therefore, $s = O(\sqrt{t})$, up to a constant of size 2-ish.
  \subsection{If you have some not-yet-specified parameter $n$ and a bunch of $\pm$ coins, and $t = \epsilon s$, (an epsilon fraction of the number of flips), and you want the sum of your coins to be large with small probability in $n$, then...}
  Here, substitute:
  \begin{enumerate}
    \item $L=1$
    \item $t = \epsilon s$
    \item $variance-thing = s$.
  \end{enumerate}
  Then if we want the probability to be $\frac{1}{n^{O(1)}}$, noting that $n$ is a parameter completely separate from anything else we've done so far, then set:
  \begin{align}
    t^2/s = \log n
  \end{align}
  or
  \begin{align}
    t^2/s = \epsilon^2 s^2 / s = \epsilon^2 s = \log n
  \end{align}
  or $s = \frac{\log n}{\eps^2}$. As usual, the $Lt/3$ term becomes non-dominant, as it is $\epsilon s$ and is much smaller than $s$.


\end{document}
